% !Mode:: "TeX:UTF-8"
% CTeX Macro-package Needed
% Compiled with XeLaTeX

\documentclass[UTF8,notitlepage,a4paper,zihao=-4,punct=quanjiao,scheme=chinese]{ctexart}

\usepackage{xeCJK} %%% Japanese Language %%%
\setCJKmainfont{ipaexm.ttf} %%% Japanese Language %%%

\usepackage{xcolor}
%\definecolor{MSBlue}{RGB}{0,112,192}
%\newcommand{\msblue}[1]{\textcolor[RGB]{0,112,192}{#1}}
\newcommand{\red}[1]{\textcolor{red}{#1}}
\newcommand{\blue}[1]{\textcolor{blue}{#1}}

\ctexset{
%%%-----------------------%%%
section={
name = {\S\,,.},
format = {\heiti\zihao{4}},
%number = \chinese{section},
%aftername+ = \hspace{-\ccwd},
hang = false,
},
%%%-----------------------%%%
subsection={
name = {,.},
format = {\kaishu\zihao{4}},
%number = \chinese{subsection},
%aftername+ = \hspace{-\ccwd},
hang = false,
},
%%%-----------------------%%%
subsubsection={
name = {,.},
format = {\heiti\zihao{-4}},
%number = \chinese{subsection},
%aftername+ = \hspace{-\ccwd},
hang = false,
},
}

\usepackage[top=25.4mm, bottom=25.4mm, left=25.4mm, right=25.4mm]{geometry}
\renewcommand{\baselinestretch}{1.20} \normalsize

\makeatletter
\newcommand{\rmnum}[1]{\romannumeral #1}
\newcommand{\Rmnum}[1]{\expandafter\@slowromancap\romannumeral #1@}
\makeatother

\usepackage{amsmath}
\usepackage{amssymb}
\usepackage{graphicx}
\usepackage[hang,scriptsize]{subfigure}
\usepackage{multirow}
\usepackage{epstopdf}
\usepackage{enumitem}
\usepackage{cite}
\usepackage{xcolor}
\usepackage[colorlinks=true,linkcolor=blue,citecolor=blue,urlcolor=blue]{hyperref}
\usepackage{framed} % 段落文字加框 %\begin{framed}...\end{framed}
\usepackage{mdframed} % 段落文字加框 %\begin{mdframed}...\end{mdframed}


\usepackage[T1]{fontenc}
\usepackage{mathpple} %%% No \hslash %%%
%\usepackage{mathpazo}
%\usepackage{pxfonts} %%% No \mathsf %%%
%\usepackage{txfonts}
%\usepackage[garamond]{mathdesign}
%\usepackage{fouriernc}
\usepackage{mathrsfs} % 英文花式字体样式 %\mathscr{#1}
\usepackage{ulem} % 文字加波浪线等下划线 %uwave{...}


\renewcommand{\vec}[1]{\boldsymbol{#1}}
%\newcommand{\ee}{\mathrm{\boldsymbol{e}}}
%\newcommand{\ii}{\mathrm{\boldsymbol{i}}}
%\newcommand{\jj}{\mathrm{\boldsymbol{j}}}
%\newcommand{\ee}{\boldsymbol{\text{e}}}
%\newcommand{\ii}{\boldsymbol{\text{i}}}
%\newcommand{\jj}{\boldsymbol{\text{j}}}
\newcommand{\ee}{\boldsymbol{e}}
\newcommand{\ii}{\boldsymbol{i}}
\newcommand{\jj}{\boldsymbol{j}}
\newcommand{\diff}{\mathrm{d}}
\newcommand{\vnabla}{\boldsymbol{\nabla}}
%\newcommand{\boldsymbol}[1]{\boldsymbol{\mathsf{#1}}}
\newcommand{\ty}[1]{\text{\texttt{#1}}} %%% For texttt font in equation %%%


%%%%%%%%%%%%%%%%%%%%%%%%%%%%%%%%%%%%%%%%%%%%%%%%%%%%%%%%%%%%%%%%%%%%%%%%%%%%%%%%%%%%%%%%%%%%%%%%%%%%
%%%%%%%%%%%%%%%%%%%%%%%%%%%%%%%%%%%%%%%%%%%%%%%%%%%%%%%%%%%%%%%%%%%%%%%%%%%%%%%%%%%%%%%%%%%%%%%%%%%%
\begin{document}
%\kaishu
\fangsong
\pagestyle{plain}

%%%%%%%%%%%%%%%%%%%%%%%%%%%%%%%%%%%%
%%%--- Title and Instructions ---%%%
%%%%%%%%%%%%%%%%%%%%%%%%%%%%%%%%%%%%
%\begin{center} \zihao{3}\kaishu{2023--2024学年第二学期}\par \end{center}
\begin{center} \zihao{2}\kaishu{《电动力学》题库}\par \end{center}
\begin{center} \zihao{4}\kaishu{吴 \quad 克 \quad 迪}\par \end{center}
\begin{center} \vspace{-4mm}\zihao{5}\kaishu{深圳大学}\par \end{center}
%\begin{center} \zihao{5}\kaishu{开始日期:2024年07月15日 \hspace{5mm}更新日期:\today}\par \end{center}
%\begin{center} \vspace{-4mm}\zihao{5}\kaishu{更新日期:\today}\par \end{center}



%%%%%%%%%%%%%%%%%%%%%%%%%%%%%%%%%%%%%
%%--- Introduction in Sections ---%%%
%%%%%%%%%%%%%%%%%%%%%%%%%%%%%%%%%%%%%
\vspace{1mm}
\begin{quotation}\kaishu
请注意,本题库中的题目和题型仅供参考。
如果对题库内容有任何问题,请及时提出讨论。十分感谢!2025级物理师范班考试范围计算题1-7章,选择题和简答题1-9章。附加题不在题库中。
%%%------%%%
%\begin{flushright}
%——\,本UP主狗头保命
%\end{flushright}
\end{quotation}


%%%%%%%%%%%%%%%%%%%%%%%%%%%%%%%%%%%%%%%%%%%
\section{单项选择题}
%%%%%%%%%%%%%%%%%%%%%%%%%%%%%%%%%%%%%%%%%%%

%%%-----------------------%%%
%\subsection{量子力学背景知识}
%%%-----------------------%%%

\begin{enumerate}
%%%%%%%%%%%%%%%%%

\item 对于任意三个矢量 $\boldsymbol{A}$, $\boldsymbol{B}$, $\boldsymbol{C}$,以下哪个矢量三重积恒等式是正确的?
\begin{enumerate}[itemsep=0mm,label=\Alph*.]
    \item $\boldsymbol{A} \times (\boldsymbol{B} \times \boldsymbol{C}) = (\boldsymbol{A} \cdot \boldsymbol{B})\boldsymbol{C} - (\boldsymbol{A} \cdot \boldsymbol{C})\boldsymbol{B}$
    \item $\boldsymbol{A} \times (\boldsymbol{B} \times \boldsymbol{C}) = (\boldsymbol{A} \cdot \boldsymbol{C})\boldsymbol{B} - (\boldsymbol{A} \cdot \boldsymbol{B})\boldsymbol{C}$
    \item $\boldsymbol{A} \cdot (\boldsymbol{B} \times \boldsymbol{C}) = \boldsymbol{B} \cdot (\boldsymbol{A} \times \boldsymbol{C})$
    \item $\boldsymbol{A} \cdot (\boldsymbol{B} \times \boldsymbol{C}) = \boldsymbol{C} \cdot (\boldsymbol{A} \times \boldsymbol{B}) + \boldsymbol{A} \cdot (\boldsymbol{C} \times \boldsymbol{B})$
\end{enumerate}

\item 对于一个标量函数 $f(x, y, z)$,梯度 $\nabla f$ 在物理上代表的含义是:
\begin{enumerate}[itemsep=0mm,label=\Alph*.]
    \item 标量 $f$ 在空间中的最大变化率及其方向。
    \item 标量 $f$ 的通量。
    \item 矢量场 $\nabla f$ 的源。
    \item 标量 $f$ 的环量。
\end{enumerate}

\item 对于一个矢量场 $\boldsymbol{v}$,其散度 $\nabla \cdot \boldsymbol{v}$ 在物理上代表的含义是:
\begin{enumerate}[itemsep=0mm,label=\Alph*.]
    \item 矢量场在某一点旋转的程度。
    \item 矢量场在某一点的流量(通量)密度或源密度。
    \item 矢量场沿闭合路径的积分。
    \item 矢量场的最大变化率。
\end{enumerate}

\item 对于任意一个定义明确的标量场 $f$,以下哪个矢量恒等式恒成立?
\begin{enumerate}[itemsep=0mm,label=\Alph*.]
    \item $\nabla \cdot (\nabla f) = 0$
    \item $\nabla \times (\nabla f) = 0$
    \item $\nabla (\nabla \cdot f) = 0$
    \item $\nabla \times (\nabla \times f) = 0$
\end{enumerate}

\item 考虑一个单位矢量 $\hat{\boldsymbol{r}} = \boldsymbol{r}/r$,其中 $r$ 是球坐标中的径向距离。在三维空间中,以下哪个关系式是正确的?
\begin{enumerate}[itemsep=0mm,label=\Alph*.]
    \item $\nabla \cdot (\frac{\hat{\boldsymbol{r}}}{r^2}) = 0$ 
    \item $\nabla \cdot \hat{\boldsymbol{r}} = 3$
    \item $\nabla \times \hat{\boldsymbol{r}} = \frac{1}{r} \hat{\boldsymbol{\phi}}$
    \item $\nabla \cdot (\boldsymbol{r}) = 3$
\end{enumerate}

\item 根据矢量微积分的基本定理,对于任意标量函数 $f(\boldsymbol{r})$ 和任意两点 $\boldsymbol{a}$ 和 $\boldsymbol{b}$ 之间的路径 $P$,以下哪个关系式是正确的?
\begin{enumerate}[itemsep=0mm,label=\Alph*.]
    \item $\int_P (\nabla f) \cdot d\boldsymbol{l} = 0$
    \item $\int_P (\nabla f) \cdot d\boldsymbol{l} = f(\boldsymbol{b}) - f(\boldsymbol{a})$
    \item $\oint (\nabla f) \cdot d\boldsymbol{l} = f(\boldsymbol{b})$
    \item $\int_P (\nabla f) \cdot d\boldsymbol{l} = \nabla f(\boldsymbol{b}) \cdot \boldsymbol{b} - \nabla f(\boldsymbol{a}) \cdot \boldsymbol{a}$
\end{enumerate}

\item 泊松方程 $\nabla^2 V = -\rho/\epsilon_0$ 的基本解可以表示为 $V(\boldsymbol{r}) \propto 1/r$ 的叠加。这与狄拉克 $\delta$ 函数有何关系?
\begin{enumerate}[itemsep=0mm,label=\Alph*.]
    \item $\nabla^2 (1/r) = 0$
    \item $\nabla^2 (1/r) = -4\pi \delta^3(\boldsymbol{r})$
    \item $\nabla^2 (1/r)$ 在原点外是 $4\pi \delta^3(\boldsymbol{r})$
    \item $\nabla^2 (1/r) = \frac{1}{r^2}$
\end{enumerate}

\item 对于一个矢量场 $\boldsymbol{v}$,斯托克斯定理(Stokes' Theorem)描述了以下哪种关系?
\begin{enumerate}[itemsep=0mm,label=\Alph*.]
    \item $\oint_S \boldsymbol{v} \cdot d\boldsymbol{a} = \int_V (\nabla \cdot \boldsymbol{v}) d\tau$
    \item $\oint_C \boldsymbol{v} \cdot d\boldsymbol{l} = \int_S (\nabla \times \boldsymbol{v}) \cdot d\boldsymbol{a}$
    \item $\int_V \nabla f d\tau = \oint_S f d\boldsymbol{a}$
    \item $\nabla \times (\nabla \cdot \boldsymbol{v}) = 0$
\end{enumerate}

\item 空间中存在一个矢量场 $\boldsymbol{v} = y \hat{\boldsymbol{x}} + z \hat{\boldsymbol{y}} + x \hat{\boldsymbol{z}}$。计算其旋度 $\nabla \times \boldsymbol{v}$。
\begin{enumerate}[itemsep=0mm,label=\Alph*.]
    \item $\hat{\boldsymbol{x}} + \hat{\boldsymbol{y}} + \hat{\boldsymbol{z}}$
    \item $-\hat{\boldsymbol{x}} - \hat{\boldsymbol{y}} - \hat{\boldsymbol{z}}$
    \item $x + y + z$
    \item $0$
\end{enumerate}

\item 当描述一个沿着 $z$ 轴均匀分布且与 $\phi$ 角无关的物理量时,最合适的坐标系和对应的微分元素是什么?
\begin{enumerate}[itemsep=0mm,label=\Alph*.]
    \item 球坐标系;体积元素 $d\tau = r^2 \sin\theta dr d\theta d\phi$
    \item 柱坐标系;体积元素 $d\tau = s ds d\phi dz$
    \item 笛卡尔坐标系;线元素 $d\boldsymbol{l} = dx \hat{\boldsymbol{x}} + dy \hat{\boldsymbol{y}} + dz \hat{\boldsymbol{z}}$
    \item 球坐标系;面积元素 $d\boldsymbol{a} = r^2 \sin\theta d\theta d\phi \hat{\boldsymbol{r}}$
\end{enumerate}



    
 \item 在静电学中,以下哪组麦克斯韦微分方程是完全独立的、互不相关的?
\begin{enumerate}[itemsep=0mm,label=\Alph*.]
    \item $\nabla \cdot \boldsymbol{E} = \rho/\epsilon_0$ 和 $\nabla \times \boldsymbol{B} = \mu_0 \boldsymbol{J}$
    \item $\nabla \cdot \boldsymbol{E} = \rho/\epsilon_0$ 和 $\nabla \times \boldsymbol{E} = 0$
    \item $\nabla \cdot \boldsymbol{B} = 0$ 和 $\nabla \times \boldsymbol{B} = \mu_0 \boldsymbol{J}$
    \item $\nabla \times \boldsymbol{E} = 0$ 和 $\nabla \times \boldsymbol{B} = \mu_0 \boldsymbol{J}$
\end{enumerate}

\item 对于一个处于静电平衡的完美导体,以下哪一性质是错误的?
\begin{enumerate}[itemsep=0mm,label=\Alph*.]
    \item 导体内部的净电荷密度 $\rho$ 处处为零。
    \item 导体内部的电势 $V$ 处处为零。
    \item 导体表面上的电场强度 $\boldsymbol{E}$ 必须垂直于表面。
    \item 导体是一个等势体,其表面和内部的电势相等。
\end{enumerate}

\item 空间中存在一个静电场 $\boldsymbol{E}$。如果电场线在某处收敛(即电场线趋于集中),这表明该处:
\begin{enumerate}[itemsep=0mm,label=\Alph*.]
    \item 电势 $V$ 存在一个最大值。
    \item 电场强度 $\boldsymbol{E}$ 沿流管法向分量的散度为负。
    \item 存在正的净电荷 $\rho > 0$。
    \item 电势 $V$ 随空间变化率最小。
\end{enumerate}

\item 对于任意给定的电荷分布 $\rho(\boldsymbol{r})$,其产生的静电势 $V(\boldsymbol{r})$ 的形式 $V(\boldsymbol{r})=\frac{1}{4\pi\epsilon_{0}}\int\frac{\rho(\boldsymbol{r}')}{|\boldsymbol{r}-\boldsymbol{r}'|}d\tau'$ 自动满足以下哪个条件?
\begin{enumerate}[itemsep=0mm,label=\Alph*.]
    \item $\nabla \cdot \boldsymbol{E} = 0$
    \item $\nabla^2 V = 0$
    \item $\nabla \times \boldsymbol{E} = 0$
    \item $V(\boldsymbol{r})$ 只有在 $r \to \infty$ 时才为零。
\end{enumerate}

\item 静电场中,从无穷远处将电荷 $q$ 移动到电场强度为 $\boldsymbol{E}$ 的某点 $\boldsymbol{r}$ 处,所做的功 $W$ 的正确表达式是:
\begin{enumerate}[itemsep=0mm,label=\Alph*.]
    \item $W = q \int_{\infty}^{\boldsymbol{r}} \boldsymbol{E} \cdot d\boldsymbol{l}$
    \item $W = -q \int_{\infty}^{\boldsymbol{r}} \boldsymbol{E} \cdot d\boldsymbol{l}$
    \item $W = \frac{1}{2} \int \rho V d\tau$
    \item $W = \frac{q}{\epsilon_0} \oint \boldsymbol{E} \cdot d\boldsymbol{a}$
\end{enumerate}

\item 两个等量异号的点电荷 $q$ 和 $-q$ 相距 $d$ 构成电偶极子。在垂直平分线上,距离中心 $r$ 处 ($r \gg d$) 的电势 $V$ 为:
\begin{enumerate}[itemsep=0mm,label=\Alph*.]
    \item $V \propto 1/r$
    \item $V \propto 1/r^2$
    \item $V = 0$
    \item $V \propto 1/r^3$
\end{enumerate}

\item 在静电学中,如果电场 $\boldsymbol{E}$ 的表达式形式为 $\boldsymbol{E} = A (y \hat{\boldsymbol{x}} + x \hat{\boldsymbol{y}})$, 其中 $A$ 是常数,那么该电场:
\begin{enumerate}[itemsep=0mm,label=\Alph*.]
    \item 是一个可能的静电场,且其电势 $V = -A x y + \text{const}$。
    \item 不是一个可能的静电场,因为 $\nabla \cdot \boldsymbol{E} \ne 0$。
    \item 是一个可能的静电场,但它必须由自由电荷产生。
    \item 不是一个可能的静电场,因为 $\nabla \times \boldsymbol{E} \ne 0$。
\end{enumerate}

\item 麦克斯韦应力张量 $\boldsymbol{T}$(其分量为 $T_{ij}$)在静电场中的物理意义是:
\begin{enumerate}[itemsep=0mm,label=\Alph*.]
    \item 电场作用在单位体积电荷上的力。
    \item 储存在单位体积电场中的能量密度。
    \item 穿过单位面积的电场能量流。
    \item 穿过单位面积的电磁动量流或电磁力。
\end{enumerate}

\item 对于一个带电导体球壳(内半径 $a$,外半径 $b$),总电荷为 $Q$。当球壳达到静电平衡时,其体电荷密度 $\rho$ 和表面电荷密度 $\sigma$ 的分布是:
\begin{enumerate}[itemsep=0mm,label=\Alph*.]
    \item $\rho$ 均匀分布在 $a < r < b$ 区域,$\sigma = 0$。
    \item $\rho = 0$ 处处为零,$\sigma = Q/(4\pi b^2)$ 仅分布在外表面 $r=b$ 上。
    \item $\rho = 0$ 处处为零,$\sigma = Q/(4\pi a^2)$ 仅分布在内表面 $r=a$ 上。
    \item $\rho = 0$ 处处为零,$\sigma$ 分布在 $r=a$ 和 $r=b$ 两个表面上。
\end{enumerate}

\item 静电场能量 $W = \frac{\epsilon_0}{2} \int E^2 d\tau$ 公式积分的区域是:
\begin{enumerate}[itemsep=0mm,label=\Alph*.]
    \item 仅包含电荷分布的区域。
    \item 仅包含导体内部的区域。
    \item 包含电荷和电场存在的所有空间区域。
    \item 仅包含导体表面的区域。
\end{enumerate} 

\item 静电学第一唯一性定理指出,在一个由导体和电荷分布包围的体积 $V$ 内,如果已知边界 $S$ 上的什么条件,则 $V$ 内的电势 $V(\boldsymbol{r})$ 是唯一的?
\begin{enumerate}[itemsep=0mm,label=\Alph*.]
    \item 边界 $S$ 上的电荷密度 $\sigma$。
    \item 边界 $S$ 上的电势 $V$。
    \item 边界 $S$ 上的电场强度 $\boldsymbol{E}$。
    \item 边界 $S$ 上的电流密度 $\boldsymbol{J}$。
\end{enumerate}

\item 在一个没有自由电荷的空间区域内,电势 $V(\boldsymbol{r})$ 满足拉普拉斯方程 $\nabla^2 V = 0$。根据拉普拉斯方程的性质,以下哪个结论是错误的?
\begin{enumerate}[itemsep=0mm,label=\Alph*.]
    \item $V$ 不可能在区域内部取到极值。
    \item $V$ 的最大值和最小值不一定出现在边界上。
    \item 如果 $V$ 在区域边界上为常数,则 $V$ 在整个区域内部为常数。
    \item 区域内任意一点的电势等于以该点为球心、完全处于该区域内任意球面上电势的平均值。
\end{enumerate}

\item 对于一个包含自由电荷 $\rho$ 的体积 $V$,电势 $V$ 满足的微分方程是:
\begin{enumerate}[itemsep=0mm,label=\Alph*.]
    \item $\nabla \cdot \boldsymbol{E} = -\frac{\rho}{\epsilon_0}$
    \item $\nabla \times \boldsymbol{E} = 0$
    \item $\nabla^2 V = 0$
    \item $\nabla^2 V = -\frac{\rho}{\epsilon_0}$
\end{enumerate}

\item 考虑一个由多个电荷组成的系统。在远离电荷源的区域 ($r \gg$ 电荷系统尺寸) 内,电势 $V(\boldsymbol{r})$ 的渐近展开式中,第一项非零贡献(单极项)与什么成正比?
\begin{enumerate}[itemsep=0mm,label=\Alph*.]
    \item 电偶极矩 $\boldsymbol{p}$
    \item 系统总电荷 $Q$
    \item 四极矩张量 $\boldsymbol{Q}$
    \item $1/r^3$
\end{enumerate}

\item 如果电势 $V(\boldsymbol{r})$ 在某一区域内满足 $\nabla^2 V = 0$,并且在边界 $S$ 上已知 $\frac{\partial V}{\partial n}$(电场 $\boldsymbol{E}$ 的法向分量),根据静电学第二唯一性定理,要保证解 $V(\boldsymbol{r})$ 唯一确定,还需要知道什么?
\begin{enumerate}[itemsep=0mm,label=\Alph*.]
    \item 边界 $S$ 上的电势 $V$ 必须为零。
    \item 区域内的总电荷必须为零。
    \item 区域内的电势 $V$ 必须在某个参考点上被指定。
    \item 边界 $S$ 上的切向电场 $\boldsymbol{E}_{||}$。
\end{enumerate}

\item 静电场中,电场强度 $\boldsymbol{E}$ 与电势 $V$ 之间的基本关系是:
\begin{enumerate}[itemsep=0mm,label=\Alph*.]
    \item $\boldsymbol{E} = -\nabla V$
    \item $\boldsymbol{E} = \nabla \cdot V$
    \item $V = -\int \boldsymbol{E} \cdot d\boldsymbol{l}$
    \item $\boldsymbol{E} = \nabla \times V$
\end{enumerate}

\item 静电势 $V$ 的基本表达式 $V(\boldsymbol{r})=\frac{1}{4\pi\epsilon_{0}}\int\frac{\rho(\boldsymbol{r}')}{|\boldsymbol{r}-\boldsymbol{r}'|}d\tau'$ 满足以下哪个物理原理?
\begin{enumerate}[itemsep=0mm,label=\Alph*.]
    \item 能量守恒定律
    \item 动量守恒定律
    \item 毕奥-萨伐尔定律
    \item 线性叠加原理
\end{enumerate}

\item 在多极展开中,如果电荷系统的单极矩 $Q=0$,但偶极矩 $\boldsymbol{p} \ne 0$,则在远处 ($r$),电势 $V(\boldsymbol{r})$ 主要由哪一项决定?
\begin{enumerate}[itemsep=0mm,label=\Alph*.]
    \item $V(\boldsymbol{r}) \propto 1/r$
    \item $V(\boldsymbol{r}) \propto 1/r^2$
    \item $V(\boldsymbol{r}) \propto 1/r^3$
    \item $V(\boldsymbol{r}) \propto r$
\end{enumerate}

\item 对于一个由自由电荷分布 $\rho$ 产生的静电场,静电能 $W$ 的体积积分表达式是:
\begin{enumerate}[itemsep=0mm,label=\Alph*.]
    \item $W = \frac{\epsilon_0}{2} \int V^2 d\tau$
    \item $W = \frac{\epsilon_0}{2} \int E^2 d\tau$
    \item $W = \frac{1}{2} \int \boldsymbol{E} \cdot \boldsymbol{B} d\tau$
    \item $W = \frac{1}{2\epsilon_0} \int \rho^2 d\tau$
\end{enumerate}

\item 镜像法能够求解电势问题的条件是:
\begin{enumerate}[itemsep=0mm,label=\Alph*.]
    \item 仅适用于无限大接地导体平面。
    \item 仅适用于电荷分布与导体相距无穷远的情况。
    \item 必须找到一组替代电荷,使其在原问题导体边界上产生的电势满足原问题的边界条件。
    \item 镜像电荷必须与原电荷具有相同的大小和符号。
\end{enumerate}

\item 在一个被极化的电介质中,极化强度 $\boldsymbol{P}$ 的定义是:
\begin{enumerate}[itemsep=0mm,label=\Alph*.]
    \item 单位体积内净束缚电荷的总量。
    \item 物质中自由电荷密度 $\rho_f$ 的体积分。
    \item 物质中电偶极矩的总和。
    \item 单位体积内的净电偶极矩。
\end{enumerate}

\item 某电介质的极化强度为 $\boldsymbol{P}$。其体束缚电荷密度 $\rho_b$ 和表面束缚电荷密度 $\sigma_b$ 的正确表达式是:
\begin{enumerate}[itemsep=0mm,label=\Alph*.]
    \item $\rho_b = \nabla \cdot \boldsymbol{P}$, $\sigma_b = \boldsymbol{P} \times \hat{\boldsymbol{n}}$
    \item $\rho_b = -\nabla \cdot \boldsymbol{P}$, $\sigma_b = \boldsymbol{P} \cdot \hat{\boldsymbol{n}}$
    \item $\rho_b = -\nabla \cdot \boldsymbol{P}$, $\sigma_b = \boldsymbol{P} \times \hat{\boldsymbol{n}}$
    \item $\rho_b = \nabla \cdot \boldsymbol{P}$, $\sigma_b = \boldsymbol{P} \cdot \hat{\boldsymbol{n}}$
\end{enumerate}

\item 辅助电场 $\boldsymbol{D}$ 场与电场强度 $\boldsymbol{E}$ 场和极化强度 $\boldsymbol{P}$ 场的基本关系是:
\begin{enumerate}[itemsep=0mm,label=\Alph*.]
    \item $\boldsymbol{D} = \epsilon_0 \boldsymbol{E} - \boldsymbol{P}$
    \item $\boldsymbol{D} = \boldsymbol{E} + \epsilon_0 \boldsymbol{P}$
    \item $\boldsymbol{D} = \epsilon_0 (\boldsymbol{E} + \boldsymbol{P})$
    \item $\boldsymbol{D} = \epsilon_0 \boldsymbol{E} + \boldsymbol{P}$
\end{enumerate}

\item 边界条件下,以下哪个量在两种电介质的交界面上,其法向分量是连续的?
\begin{enumerate}[itemsep=0mm,label=\Alph*.]
    \item 电场强度 $\boldsymbol{E}$
    \item 极化强度 $\boldsymbol{P}$
    \item 辅助电场 $\boldsymbol{D}$ (假设交界面上没有自由表面电荷 $\sigma_f$)
    \item 电势 $V$ 的法向导数
\end{enumerate}

\item 麦克斯韦方程组中,关于辅助电场 $\boldsymbol{D}$ 的高斯定律微分形式是:
\begin{enumerate}[itemsep=0mm,label=\Alph*.]
    \item $\nabla \cdot \boldsymbol{D} = 0$
    \item $\nabla \cdot \boldsymbol{D} = \rho_b$
    \item $\nabla \cdot \boldsymbol{D} = \rho_f$
    \item $\nabla \cdot \boldsymbol{D} = \rho_f + \rho_b$
\end{enumerate}

\item 对于一个各向同性的线性电介质,其相对介电常数 $\epsilon_r$ 和电极化率 $\chi_e$ 之间的关系是:
\begin{enumerate}[itemsep=0mm,label=\Alph*.]
    \item $\epsilon_r = 1 / \chi_e$
    \item $\epsilon_r = 1 + \chi_e$
    \item $\chi_e = \epsilon_0 (\epsilon_r - 1)$
    \item $\epsilon_r = \epsilon_0 \chi_e$
\end{enumerate}

\item 将一个均匀极化的无限大平板插入一个由自由电荷产生的均匀电场 $\boldsymbol{E}_0$ 中,如果平板的极化方向与 $\boldsymbol{E}_0$ 方向相同,则平板内部的总电场 $\boldsymbol{E}$ 的大小:
\begin{enumerate}[itemsep=0mm,label=\Alph*.]
    \item 等于 $\boldsymbol{E}_0$
    \item 大于 $\boldsymbol{E}_0$
    \item 小于 $\boldsymbol{E}_0$
    \item $\boldsymbol{E}_0 + \boldsymbol{P} / \epsilon_0$
\end{enumerate}

\item 如果电介质 1 (介电常数 $\epsilon_1$) 和电介质 2 (介电常数 $\epsilon_2$) 在交界面上无自由电荷,$\boldsymbol{E}_1$ 与界面法线 $\hat{\boldsymbol{n}}$ (指向介质 2) 的夹角为 $\theta_1$。则 $\boldsymbol{E}_1$ 与 $\boldsymbol{E}_2$ 之间的折射定律为:
\begin{enumerate}[itemsep=0mm,label=\Alph*.]
    \item $\epsilon_1 \tan\theta_1 = \epsilon_2 \tan\theta_2$
    \item $\epsilon_1 \sin\theta_1 = \epsilon_2 \sin\theta_2$
    \item  $\frac{\epsilon_1}{\epsilon_2} = \frac{\tan\theta_1}{\tan\theta_2} $    
    \item $\frac{\epsilon_2}{\epsilon_1}= \frac{\cos\theta_2}{\cos\theta_1}$
\end{enumerate}



\item 电介质中静电能的能量密度 $u$ 的正确表达式(适用于线性电介质)是:
\begin{enumerate}[itemsep=0mm,label=\Alph*.]
    \item $u = \frac{1}{2} \epsilon_0 E^2$
    \item $u = \frac{1}{2} \boldsymbol{E} \cdot \boldsymbol{P}$
    \item $u = \frac{1}{2} \boldsymbol{D} \cdot \boldsymbol{E}$
    \item $u = \frac{1}{2} \boldsymbol{E} \cdot (\boldsymbol{D} + \boldsymbol{P})$
\end{enumerate}

\item 在一个非均匀极化的电介质中,其内部的束缚电荷 $\rho_b$ 的存在意味着:
\begin{enumerate}[itemsep=0mm,label=\Alph*.]
    \item 介质内部存在自由电荷。
    \item 介质内部的电场 $\boldsymbol{E}$ 恒为零。
    \item 介质内部的辅助电场 $\boldsymbol{D}$ 恒为零。
    \item 介质内部的净电荷密度 ($\rho_f + \rho_b$) 不为零。
\end{enumerate}
\item 静磁学中,磁感应强度 $\boldsymbol{B}$ 场满足的两个基本麦克斯韦方程是:
\begin{enumerate}[itemsep=0mm,label=\Alph*.]
    \item $\nabla \cdot \boldsymbol{B} = 0$ 和 $\nabla \times \boldsymbol{B} = \mu_0 \boldsymbol{J}$
    \item $\nabla \cdot \boldsymbol{B} = \rho / \epsilon_0$ 和 $\nabla \times \boldsymbol{B} = 0$
    \item $\nabla \cdot \boldsymbol{B} = 0$ 和 $\nabla \times \boldsymbol{B} = \mu_0 \boldsymbol{J} + \mu_0 \epsilon_0 \frac{\partial \boldsymbol{E}}{\partial t}$
    \item $\nabla \cdot \boldsymbol{B} = \rho_m / \mu_0$ 和 $\nabla \times \boldsymbol{B} = \boldsymbol{J}$
\end{enumerate}

\item 磁矢势 $\boldsymbol{A}$ 的定义 $\boldsymbol{B} = \nabla \times \boldsymbol{A}$ 保证了哪个基本定理的自然成立?
\begin{enumerate}[itemsep=0mm,label=\Alph*.]
    \item 高斯磁定律 $\nabla \cdot \boldsymbol{B} = 0$
    \item 毕奥-萨伐尔定律
    \item 麦克斯韦-安培定律
    \item 动量守恒定律
\end{enumerate}

\item 在静磁学中,磁矢势 $\boldsymbol{A}$ 在库仑规范 ($\nabla \cdot \boldsymbol{A} = 0$) 下满足的微分方程是:
\begin{enumerate}[itemsep=0mm,label=\Alph*.]
    \item $\nabla^2 \boldsymbol{A} = -\frac{1}{\mu_0} \boldsymbol{J}$
    \item $\nabla^2 \boldsymbol{A} = -\mu_0 \boldsymbol{J}$
    \item $\nabla^2 \boldsymbol{A} = 0$
    \item $\nabla^2 \boldsymbol{A} - \frac{1}{c^2} \frac{\partial^2 \boldsymbol{A}}{\partial t^2} = -\mu_0 \boldsymbol{J}$
\end{enumerate}

\item 考虑一个无限长直导线上的稳恒电流 $I$,其磁场 $\boldsymbol{B}$ 的方向由安培定律确定。如果电流方向反向,那么 $\boldsymbol{B}$ 场将发生什么变化?
\begin{enumerate}[itemsep=0mm,label=\Alph*.]
    \item $\boldsymbol{B}$ 的大小和方向都不变。
    \item $\boldsymbol{B}$ 的方向反向,大小减半。
    \item $\boldsymbol{B}$ 的方向反向,大小不变。
    \item $\boldsymbol{B}$ 的大小减半,方向不变。
\end{enumerate}

\item 洛伦兹力对作匀速圆周运动的带电粒子做功是多少?
\begin{enumerate}[itemsep=0mm,label=\Alph*.]
    \item $q \boldsymbol{E} \cdot \boldsymbol{v}$
    \item $q (\boldsymbol{v} \times \boldsymbol{B}) \cdot \boldsymbol{v}$
    \item 零
    \item $\boldsymbol{m} \cdot \boldsymbol{B}$
\end{enumerate}

\item 对于一个空间任意分布的稳恒电流 $\boldsymbol{J}$,在远离电流源的区域 ($r \gg$ 电流源尺寸) 内,磁矢势 $\boldsymbol{A}(\boldsymbol{r})$ 的第一项非零贡献(磁偶极项)与什么量成正比?
\begin{enumerate}[itemsep=0mm,label=\Alph*.]
    \item $1/r$
    \item $1/r^2$
    \item $1/r^3$
    \item $\boldsymbol{r}$
\end{enumerate}

\item 磁偶极矩 $\boldsymbol{m}$ 在均匀磁场 $\boldsymbol{B}$ 中所受的力 $\boldsymbol{F}$ 和力矩 $\boldsymbol{N}$ 分别为:
\begin{enumerate}[itemsep=0mm,label=\Alph*.]
    \item $\boldsymbol{F} = \boldsymbol{m} \times \boldsymbol{B}$, $\boldsymbol{N} = \nabla(\boldsymbol{m} \cdot \boldsymbol{B})$
    \item $\boldsymbol{F} = 0$, $\boldsymbol{N} = \boldsymbol{m} \times \boldsymbol{B}$
    \item $\boldsymbol{F} = \nabla(\boldsymbol{m} \cdot \boldsymbol{B})$, $\boldsymbol{N} = 0$
    \item $\boldsymbol{F} = 0$, $\boldsymbol{N} = \boldsymbol{m} \cdot \boldsymbol{B}$
\end{enumerate}

\item 安培定律 $\oint \boldsymbol{B} \cdot d\boldsymbol{l} = \mu_0 I_{\text{enc}}$ 的有效性条件是:
\begin{enumerate}[itemsep=0mm,label=\Alph*.]
    \item 只有在电流是恒定且均匀时才成立。
    \item 必须是稳恒电流,即 $\frac{\partial \rho}{\partial t} = 0$。
    \item 必须是稳恒电流,且积分路径必须是圆形。
    \item 在任何有电流的系统中都成立。
\end{enumerate}

\item 在一个电流分布的边界上,磁矢势 $\boldsymbol{A}$ 的哪个分量是连续的?(假设电流密度 $\boldsymbol{J}$ 是有限的)
\begin{enumerate}[itemsep=0mm,label=\Alph*.]
    \item 只有法向分量 $\boldsymbol{A}^\perp$
    \item 只有切向分量 $\boldsymbol{A}^{||}$
    \item 法向和切向分量都连续
    \item 只有在 $\boldsymbol{J}=0$ 时才连续
\end{enumerate}

\item 根据毕奥-萨伐尔定律,一个无限小电流元 $I d\boldsymbol{l}$ 在空间产生的磁场 $d\boldsymbol{B}$ 的方向总是:
\begin{enumerate}[itemsep=0mm,label=\Alph*.]
    \item 沿着电流元 $d\boldsymbol{l}$ 的方向。
    \item 沿着电流元 $d\boldsymbol{l}$ 与位置矢量 $\hat{\boldsymbol{r}}$ 的叉积方向。
    \item 沿着位置矢量 $\hat{\boldsymbol{r}}$ 的方向。
    \item 垂直于 $d\boldsymbol{l}$ 且指向 $d\boldsymbol{l}$。
\end{enumerate}

\item 在一个被磁化的物质中,磁化强度 $\boldsymbol{M}$ 的定义是:
\begin{enumerate}[itemsep=0mm,label=\Alph*.]
    \item 物质中自由电流密度 $\boldsymbol{J}_f$ 的体积分。
    \item 物质中磁偶极矩的总和。
    \item 单位体积内的净磁偶极矩。
    \item 辅助磁场 $\boldsymbol{H}$ 的旋度。
\end{enumerate}

\item 对于一个均匀磁化的有限物体,其体束缚电流密度 $\boldsymbol{J}_b$ 的表达式 $\boldsymbol{J}_b = \nabla \times \boldsymbol{M}$ 恒为:
\begin{enumerate}[itemsep=0mm,label=\Alph*.]
    \item 零
    \item $\boldsymbol{M} \times \hat{\boldsymbol{n}}$
    \item $\boldsymbol{M}$ 的梯度
    \item $-\frac{\partial \boldsymbol{M}}{\partial t}$
\end{enumerate}

\item 磁场强度 $\boldsymbol{H}$ 场与磁感应强度 $\boldsymbol{B}$ 场的基本关系是:
\begin{enumerate}[itemsep=0mm,label=\Alph*.]
    \item $\boldsymbol{B} = \mu_0 (\boldsymbol{H} - \boldsymbol{M})$
    \item $\boldsymbol{H} = \mu_0 \boldsymbol{B} + \boldsymbol{M}$
    \item $\boldsymbol{B} = \mu_0 (\boldsymbol{H} + \boldsymbol{M})$
    \item $\boldsymbol{B} = \epsilon_0 \boldsymbol{E} + \boldsymbol{P}$
\end{enumerate}

\item 麦克斯韦方程组中,关于 $\boldsymbol{H}$ 场旋度的方程是:
\begin{enumerate}[itemsep=0mm,label=\Alph*.]
    \item $\nabla \times \boldsymbol{H} = \boldsymbol{J}_f + \frac{\partial \boldsymbol{D}}{\partial t}$
    \item $\nabla \times \boldsymbol{H} = \boldsymbol{J}_f$
    \item $\nabla \times \boldsymbol{H} = \boldsymbol{J}_b$
    \item $\nabla \times \boldsymbol{H} = \boldsymbol{J}_f - \boldsymbol{J}_b$
\end{enumerate}

\item 边界条件下,以下哪个量在两种介质的交界面上,其切向分量是连续的?
\begin{enumerate}[itemsep=0mm,label=\Alph*.]
    \item 磁化强度 $\boldsymbol{M}$
    \item 束缚电流密度 $\boldsymbol{K}_b$
    \item 磁感应强度 $\boldsymbol{B}$
    \item 辅助磁场 $\boldsymbol{H}$ (假设交界面上没有自由表面电流 $\boldsymbol{K}_f$)
\end{enumerate}

\item 对于一个各向同性的线性磁介质,磁化强度 $\boldsymbol{M}$ 与磁场强度 $\boldsymbol{H}$ 的关系是 $\boldsymbol{M} = \chi_m \boldsymbol{H}$,其中 $\chi_m$ 为磁化率。如果该介质是抗磁性的,那么 $\chi_m$ 的数值特性是:
\begin{enumerate}[itemsep=0mm,label=\Alph*.]
    \item $\chi_m \gg 1$
    \item $\chi_m < 0$ 且 $| \chi_m | \ll 1$
    \item $\chi_m > 0$ 且 $| \chi_m | \ll 1$
    \item $\chi_m = 0$
\end{enumerate}

\item 在一个均匀磁化的长螺线管内部,如果其磁化强度 $\boldsymbol{M}$ 沿螺线管轴向,则其内部的束缚电流分布为:
\begin{enumerate}[itemsep=0mm,label=\Alph*.]
    \item 均匀分布的体束缚电流 $\boldsymbol{J}_b$。
    \item 仅存在表面束缚电流 $\boldsymbol{K}_b$。
    \item 均匀分布的体束缚电流 $\boldsymbol{J}_b$ 和表面束缚电流 $\boldsymbol{K}_b$。
    \item 内部磁场 $\boldsymbol{B}$ 恒为零。
\end{enumerate}

\item 某线性磁介质的相对磁导率 $\mu_r = 1000$。如果将该介质视为真空 ($\mu_0$) 和磁化强度 $\boldsymbol{M}$ 的叠加效应,那么 $\boldsymbol{M}$ 对 $\boldsymbol{B}$ 的贡献大约是总 $\boldsymbol{B}$ 场的:
\begin{enumerate}[itemsep=0mm,label=\Alph*.]
    \item $0.1\%$
    \item $1\%$
    \item $99.9\%$
    \item $1000$ 倍
\end{enumerate}

\item 经典电动力学认为所有磁现象均源于电流(安培观点)。引入 $\boldsymbol{M}$ 和 $\boldsymbol{H}$ 的主要物理动机是:
\begin{enumerate}[itemsep=0mm,label=\Alph*.]
    \item 简化麦克斯韦方程组的代数形式。
    \item 将物质的响应(束缚电流)从方程中分离出来,使方程仅由自由电流 $\boldsymbol{J}_f$ 决定。
    \item 严格区分抗磁性和顺磁性物质。
    \item 为磁单极子的存在提供理论基础。
\end{enumerate}

\item 在一个均匀的线性磁介质中,由静自由电流 $\boldsymbol{J}_f$ 产生的磁感应强度 $\boldsymbol{B}$ 场,与真空情况下的 $\boldsymbol{B}_{vac}$ 场之间的关系是:
\begin{enumerate}[itemsep=0mm,label=\Alph*.]
    \item $\boldsymbol{B} = \mu_r \boldsymbol{B}_{vac}$
    \item $\boldsymbol{B} = \frac{1}{\mu_r} \boldsymbol{B}_{vac}$
    \item $\boldsymbol{B} = \boldsymbol{B}_{vac} + \mu_0 \boldsymbol{M}$
    \item $\boldsymbol{B}$ 与 $\boldsymbol{B}_{vac}$ 之间没有简单的比例关系。
\end{enumerate}

\item 在一般(非静止)介质中,描述电流密度 $\boldsymbol{J}$ 与电磁场 $\boldsymbol{E}$ 和 $\boldsymbol{B}$ 关系的欧姆定律(微分形式)是:
\begin{enumerate}[itemsep=0mm,label=\Alph*.]
    \item $\boldsymbol{J} = \sigma \boldsymbol{E}$
    \item $\boldsymbol{J} = \sigma (\boldsymbol{E} + \boldsymbol{v} \times \boldsymbol{B})$
    \item $\boldsymbol{J} = \boldsymbol{E} / \rho$
    \item $\boldsymbol{J} = \epsilon_0 \frac{\partial \boldsymbol{E}}{\partial t}$
\end{enumerate}

\item 连续性方程(电荷守恒定律的微分形式)在电动力学中是:
\begin{enumerate}[itemsep=0mm,label=\Alph*.]
    \item $\nabla \cdot \boldsymbol{J} = 0$
    \item $\nabla \cdot \boldsymbol{J} = \frac{\partial \rho}{\partial t}$
    \item $\nabla \cdot \boldsymbol{J} = - \frac{\partial \rho}{\partial t}$
    \item $\nabla \times \boldsymbol{J} = - \frac{\partial \rho}{\partial t}$
\end{enumerate}

\item 麦克斯韦修正项(位移电流密度 $\boldsymbol{J}_D$)的引入,使得修正后的安培定律 $\nabla \times \boldsymbol{B} = \mu_0 (\boldsymbol{J} + \boldsymbol{J}_D)$ 的散度$\nabla \cdot (\nabla \times \boldsymbol{B})$ 满足恒等式 $\nabla \cdot (\nabla \times \boldsymbol{B}) = 0$。这等价于要求:
\begin{enumerate}[itemsep=0mm,label=\Alph*.]
    \item $\nabla \cdot \boldsymbol{B} = 0$
    \item $\nabla \cdot \boldsymbol{E} = 0$
    \item $\nabla \cdot \boldsymbol{J}_{total} = 0$
    \item $\nabla \times \boldsymbol{E} = 0$
\end{enumerate}

\item 法拉第电磁感应定律的微分形式是:
\begin{enumerate}[itemsep=0mm,label=\Alph*.]
    \item $\nabla \cdot \boldsymbol{E} = \rho / \epsilon_0$
    \item $\nabla \times \boldsymbol{E} = - \frac{\partial \boldsymbol{B}}{\partial t}$
    \item $\nabla \cdot \boldsymbol{B} = 0$
    \item $\nabla \times \boldsymbol{H} = \boldsymbol{J} + \frac{\partial \boldsymbol{D}}{\partial t}$
\end{enumerate}

\item 在一个线性、均匀、各向同性、无源($\rho_f=0$)的非导电介质中($\sigma=0$),麦克斯韦方程组的安培-麦克斯韦定律微分形式简化为:
\begin{enumerate}[itemsep=0mm,label=\Alph*.]
    \item $\nabla \times \boldsymbol{H} = \boldsymbol{J}_f$
    \item $\nabla \times \boldsymbol{B} = \mu \epsilon \frac{\partial \boldsymbol{E}}{\partial t}$
    \item $\nabla \times \boldsymbol{H} = 0$
    \item $\nabla \times \boldsymbol{B} = \mu_0 \boldsymbol{J}$
\end{enumerate}

\item 在均匀导电介质中,自由电荷密度的衰减方程 $\rho(t) = \rho(0) e^{-t/\tau}$ 中的弛豫时间 $\tau$ 表达式为:
\begin{enumerate}[itemsep=0mm,label=\Alph*.]
    \item $\tau = \epsilon_0 / \sigma$
    \item $\tau = \sigma / \epsilon_0$
    \item $\tau = \epsilon / \sigma$
    \item $\tau = \sigma \epsilon$
\end{enumerate}

\item 考虑一个由理想导体构成的闭合回路在时变磁场中运动。回路中的总电动势 $\mathcal{E}_{total}$ 是多少?
\begin{enumerate}[itemsep=0mm,label=\Alph*.]
    \item $\mathcal{E}_{total} = \oint (\boldsymbol{E} + \boldsymbol{v} \times \boldsymbol{B}) \cdot d\boldsymbol{l}$
    \item $\mathcal{E}_{total} = - \frac{d\Phi_B}{dt}$
    \item $\mathcal{E}_{total} = 0$
    \item $\mathcal{E}_{total} = \oint \boldsymbol{E} \cdot d\boldsymbol{l}$
\end{enumerate}

\item 在麦克斯韦方程组中,哪个定律的微分形式不能直接从某个矢量场的散度恒为零或旋度恒为零的恒等式得到?
\begin{enumerate}[itemsep=0mm,label=\Alph*.]
    \item 高斯电场定律($\nabla \cdot \boldsymbol{E} = \rho / \epsilon_0$)
    \item 磁场的高斯定律($\nabla \cdot \boldsymbol{B} = 0$)
    \item 法拉第电磁感应定律($\nabla \times \boldsymbol{E} = - \frac{\partial \boldsymbol{B}}{\partial t}$)
    \item 安培-麦克斯韦定律($\nabla \times \boldsymbol{B} = \mu_0 \boldsymbol{J} + \mu_0 \epsilon_0 \frac{\partial \boldsymbol{E}}{\partial t}$)
\end{enumerate}

\item 动生电动势 $\mathcal{E}_{motional}$ 的物理来源是:
\begin{enumerate}[itemsep=0mm,label=\Alph*.]
    \item 随时间变化的磁场产生的感生电场 $\boldsymbol{E}_{induce}$。
    \item 导体内部自由电荷所受的洛伦兹磁力 $\boldsymbol{F}_m = q (\boldsymbol{v} \times \boldsymbol{B})$。
    \item 导体内部的静电场 $\boldsymbol{E}_{static}$。
    \item 导体运动产生的位移电流。
\end{enumerate}

\item 麦克斯韦对安培定律的修正(引入位移电流)表明,在时变场中,磁场的旋度 $\nabla \times \boldsymbol{B}$ 是由以下哪两项共同决定的?
\begin{enumerate}[itemsep=0mm,label=\Alph*.]
    \item 仅由传导电流 $\boldsymbol{J}$。
    \item 仅由位移电流 $\boldsymbol{J}_D$。
    \item 传导电流 $\boldsymbol{J}$ 和磁通量变化率 $\frac{\partial \boldsymbol{B}}{\partial t}$。
    \item 传导电流 $\boldsymbol{J}$ 和电场变化率 $\frac{\partial \boldsymbol{E}}{\partial t}$。
\end{enumerate}

\item 对于一个电荷密度 $\rho(\boldsymbol{r}, t)$ 随时间变化但无传导电流的区域 ($\boldsymbol{J}=0$),根据连续性方程和高斯定律,$\nabla \cdot (\epsilon_0 \frac{\partial \boldsymbol{E}}{\partial t})$ 应该等于:
\begin{enumerate}[itemsep=0mm,label=\Alph*.]
    \item $0$
    \item $\frac{\partial \rho}{\partial t}$
    \item $- \frac{\partial \rho}{\partial t}$
    \item $\rho / \epsilon_0$
\end{enumerate}

\item 自感系数 $L$ 的定义式 $\Phi = L I$ 中的磁通量 $\Phi$ 是指:
\begin{enumerate}[itemsep=0mm,label=\Alph*.]
    \item 产生电流 $I$ 的外部磁场通过电路的磁通量。
    \item 由电流 $I$ 自身产生的磁场通过电路的磁通量。
    \item 仅与电路几何形状和磁导率有关的常数。
    \item 外部磁场和自身磁场通过电路的总磁通量。
\end{enumerate}

\item 如果一个无限大理想导体的表面电荷密度 $\sigma_0$ 发生变化,电荷在导体内部弛豫的速度:
\begin{enumerate}[itemsep=0mm,label=\Alph*.]
    \item 非常慢,因为 $\sigma \to \infty$ 导致 $\tau \to 0$。
    \item 瞬间完成,因为 $\sigma \to \infty$ 导致 $\tau \to 0$。
    \item 非常慢,因为 $\sigma \to \infty$ 导致 $\tau \to \infty$。
    \item 无法确定,因为理想导体内部没有电荷。
\end{enumerate}

\item 在电磁感应现象中,楞次定律本质上是哪个基本物理定律在电磁学中的体现?
\begin{enumerate}[itemsep=0mm,label=\Alph*.]
    \item 电荷守恒定律
    \item 动量守恒定律
    \item 能量守恒定律
    \item 角动量守恒定律
\end{enumerate}

\item 麦克斯韦方程组的完整微分形式(在真空中)共有几个独立的方程?
\begin{enumerate}[itemsep=0mm,label=\Alph*.]
    \item 2 个
    \item 4 个
    \item 6 个
    \item 8 个
\end{enumerate}

\item 对于一个具有均匀电导率 $\sigma$ 和介电常数 $\epsilon$ 的介质,其内部的传导电流密度 $\boldsymbol{J}$ 与位移电流密度 $\boldsymbol{J}_D$ 之比 $|\boldsymbol{J} / \boldsymbol{J}_D|$ 的量级关系决定了该介质对电磁波的响应。如果 $|\boldsymbol{J} / \boldsymbol{J}_D| \gg 1$,该介质可被视为:
\begin{enumerate}[itemsep=0mm,label=\Alph*.]
    \item 理想介质(非导电)
    \item 理想导体
    \item 具有良好导电性的良导体
    \item 磁性物质
\end{enumerate}

\item 根据法拉第电磁感应定律的微分形式 $\nabla \times \boldsymbol{E} = - \frac{\partial \boldsymbol{B}}{\partial t}$,以下哪个结论是错误的?
\begin{enumerate}[itemsep=0mm,label=\Alph*.]
    \item 时变的磁场产生感生电场。
    \item 感生电场是一个无旋场。
    \item 感生电场是一个非保守场。
    \item 感生电场的场线可以是闭合的。
\end{enumerate}

\item 耦合电路中,互感系数 $M_{21}$ 与 $M_{12}$ 的关系是:
\begin{enumerate}[itemsep=0mm,label=\Alph*.]
    \item $M_{21} = M_{12}$
    \item $M_{21} = - M_{12}$
    \item $M_{21} = M_{12}^{-1}$
    \item $M_{21} \ne M_{12}$
\end{enumerate}

\item 麦克斯韦方程组中的 $\nabla \cdot \boldsymbol{B} = 0$ 和 $\nabla \times \boldsymbol{E} = - \frac{\partial \boldsymbol{B}}{\partial t}$ 两个方程,使得电场 $\boldsymbol{E}$ 和磁场 $\boldsymbol{B}$ 可以通过哪个势函数来统一描述?
\begin{enumerate}[itemsep=0mm,label=\Alph*.]
    \item 仅标量势 $V$
    \item 仅矢量势 $\boldsymbol{A}$
    \item 标量势 $V$ 和矢量势 $\boldsymbol{A}$
    \item 仅磁化矢量 $\boldsymbol{M}$
\end{enumerate}

\item 假设一个导体内自由电荷的运动满足 $\boldsymbol{J} = \sigma \boldsymbol{E}$。如果此导体处于一个电场 $\boldsymbol{E}(\boldsymbol{r}, t)$ 中,电场强度 $\boldsymbol{E}$ 满足哪个方程?(忽略 $\boldsymbol{B}$ 场的影响)
\begin{enumerate}[itemsep=0mm,label=\Alph*.]
    \item $\nabla^2 \boldsymbol{E} = \mu_0 \epsilon_0 \frac{\partial^2 \boldsymbol{E}}{\partial t^2}$
    \item $\nabla^2 \boldsymbol{E} = \mu_0 \sigma \frac{\partial \boldsymbol{E}}{\partial t}$
    \item $\nabla^2 \boldsymbol{E} = \mu_0 \sigma \frac{\partial \boldsymbol{E}}{\partial t} + \mu_0 \epsilon_0 \frac{\partial^2 \boldsymbol{E}}{\partial t^2}$
    \item $\nabla \cdot \boldsymbol{E} = 0$
\end{enumerate}

\item 连续性方程 $\frac{\partial \rho}{\partial t} = -\nabla \cdot \boldsymbol{J}$ 表达了电荷的局域守恒。它在电动力学中是:
\begin{enumerate}[itemsep=0mm,label=\Alph*.]
    \item 独立于麦克斯韦方程组的基本假设。
    \item 麦克斯韦方程组的推论,但不要求电场和磁场必须同时存在。
    \item 只有在 $\nabla \times \boldsymbol{H} = \boldsymbol{J}$ 的稳恒电流条件下才成立。
    \item 麦克斯韦方程组的固有属性,可通过对安培-麦克斯韦定律取散度并结合高斯定律推导得出。
\end{enumerate}

\item 波印廷定理 $\frac{dW}{dt}=-\frac{d}{dt}\int_{\mathcal{V}}u~d\tau-\oint_{\mathcal{S}}\boldsymbol{S}\cdot da$ 中的 $\frac{dW}{dt}$ 项代表的物理量是:
\begin{enumerate}[itemsep=0mm,label=\Alph*.]
    \item 电磁场对体元 $\mathcal{V}$ 内电荷做功的功率。
    \item 储存在体元 $\mathcal{V}$ 内的电磁场能量的增加率。
    \item 通过表面 $\mathcal{S}$ 流出体元 $\mathcal{V}$ 的电磁能量。
    \item 电荷产生的机械功与电磁场功之和。
\end{enumerate}

\item 在一个没有自由电荷和电流 ($\rho=0, \boldsymbol{J}=0$) 的区域内,电磁能量的守恒形式为:
\begin{enumerate}[itemsep=0mm,label=\Alph*.]
    \item $\nabla \cdot \boldsymbol{S} = 0$
    \item $\frac{\partial u}{\partial t} = 0$
    \item $\frac{\partial u}{\partial t} = -\frac{1}{\mu_0} \nabla \cdot (\boldsymbol{E} \times \boldsymbol{B})$
    \item $\frac{\partial u}{\partial t} = -\nabla \cdot \boldsymbol{S}$
\end{enumerate}

\item 波印廷矢量 $\boldsymbol{S} \equiv \frac{1}{\mu_0}(\boldsymbol{E} \times \boldsymbol{B})$ 的物理意义是:
\begin{enumerate}[itemsep=0mm,label=\Alph*.]
    \item 空间中电磁场的能量密度。
    \item 垂直于 $\boldsymbol{E}$ 和 $\boldsymbol{B}$ 方向上的电磁场力。
    \item 单位时间内、单位面积穿过表面积的电磁能量流密度(能流密度)。
    \item 电磁场对电荷做功的功率密度。
\end{enumerate}

\item 在电动力学中,牛顿第三定律(作用力与反作用力相等且方向相反)对于两个相互作用的运动点电荷不成立。其根本原因是:
\begin{enumerate}[itemsep=0mm,label=\Alph*.]
    \item 洛伦兹力公式本身不具有对称性。
    \item 电磁相互作用的传播速度是有限的(光速)。
    \item 电场和磁场本身携带动量,弥补了粒子动量的非守恒。
    \item 麦克斯韦方程组是线性微分方程组。
\end{enumerate}

\item 电磁场动量密度 $\boldsymbol{g}$ 的正确表达式是:
\begin{enumerate}[itemsep=0mm,label=\Alph*.]
    \item $\boldsymbol{g} = \epsilon_0 \mu_0 \boldsymbol{S}$
    \item $\boldsymbol{g} = \frac{1}{c^2} \boldsymbol{S}$
    \item $\boldsymbol{g} = \epsilon_0 (\boldsymbol{E} \times \boldsymbol{B})$
    \item 以上所有表达式均正确。
\end{enumerate}

\item 麦克斯韦应力张量 $\overleftrightarrow{T}$ 在静电场中,其对角元素 $T_{xx}$ 的表达式为:
\begin{enumerate}[itemsep=0mm,label=\Alph*.]
    \item $T_{xx} = \frac{1}{2}\epsilon_0 (E_x^2 + E_y^2 + E_z^2)$
    \item $T_{xx} = \frac{1}{2}\epsilon_0 (E_x^2 - E_y^2 - E_z^2)$
    \item $T_{xx} = \epsilon_0 E_x^2 - \frac{1}{2}\epsilon_0 E^2$
    \item $T_{xx} = \epsilon_0 E_x^2$
\end{enumerate}

\item 麦克斯韦应力张量 $\overleftrightarrow{T}$ 的非对角元素 $T_{ij}$(其中 $i \ne j$)的物理意义是:
\begin{enumerate}[itemsep=0mm,label=\Alph*.]
    \item 作用在表面上的法向压力。
    \item 作用在表面上的切向剪应力。
    \item 电磁场的能量密度。
    \item 电磁场的动量密度。
\end{enumerate}

\item 电磁场动量 $\boldsymbol{P}_{em}$ 在一个任意体积 $\mathcal{V}$ 内的局域守恒定律($\boldsymbol{f}$ 为洛伦兹力密度)可表述为:
\begin{enumerate}[itemsep=0mm,label=\Alph*.]
    \item $\frac{d\boldsymbol{P}_{mech}}{dt} = \boldsymbol{F}_{em}$
    \item $\frac{d\boldsymbol{P}_{mech}}{dt} + \frac{d\boldsymbol{P}_{em}}{dt} = 0$
    \item $\frac{d\boldsymbol{P}_{mech}}{dt} = \oint_{\mathcal{S}} \overleftrightarrow{T} \cdot d\boldsymbol{a} - \frac{d\boldsymbol{P}_{em}}{dt}$
    \item $\frac{d}{dt} \int_{\mathcal{V}} \boldsymbol{g} d\tau = \nabla \cdot \overleftrightarrow{T}$
\end{enumerate}

\item 麦克斯韦应力张量 $\overleftrightarrow{T}$ 与动量守恒的关系中,积分 $\oint_{\mathcal{S}} \overleftrightarrow{T} \cdot d\boldsymbol{a}$ 在物理上表示:
\begin{enumerate}[itemsep=0mm,label=\Alph*.]
    \item 通过表面 $\mathcal{S}$ 流出体元 $\mathcal{V}$ 的总动量。
    \item 作用在体元 $\mathcal{V}$ 内电荷上的总电磁力。
    \item 作用在体元 $\mathcal{V}$ 边界 $\mathcal{S}$ 上的电磁力。
    \item 通过表面 $\mathcal{S}$ 流入体元 $\mathcal{V}$ 的电磁动量流。
\end{enumerate}

\item 在真空中传播的单色平面电磁波,其电场 $\boldsymbol{E}$、磁场 $\boldsymbol{B}$ 和波矢量 $\boldsymbol{k}$ 之间的关系,哪个是\textbf{错误}的?
\begin{enumerate}[itemsep=0mm,label=\Alph*.]
    \item $\boldsymbol{E} \perp \boldsymbol{B}$
    \item $\boldsymbol{k} \perp \boldsymbol{E}$
    \item $|\boldsymbol{B}| = \frac{1}{c} |\boldsymbol{E}|$
    \item $\boldsymbol{E} = c(\boldsymbol{k} \times \boldsymbol{B})$
\end{enumerate}

\item 麦克斯韦方程组的推论表明,真空中 $\boldsymbol{E}$ 场满足波动方程 $\nabla^2 \boldsymbol{E} - \frac{1}{c^2} \frac{\partial^2 \boldsymbol{E}}{\partial t^2} = 0$。该波动方程成立所依赖的麦克斯韦方程是:
\begin{enumerate}[itemsep=0mm,label=\Alph*.]
    \item 法拉第感应定律和高斯定律
    \item 安培-麦克斯韦定律和高斯定律
    \item 法拉第感应定律和安培-麦克斯韦定律
    \item 法拉第感应定律和磁场无源定律
\end{enumerate}

\item 对于一个在真空中沿 $z$ 方向传播的单色平面电磁波,其 $\boldsymbol{E}$ 场的振幅为 $\boldsymbol{E}_0 = E_{0x} \hat{\boldsymbol{x}} + i E_{0y} \hat{\boldsymbol{y}}$。该电磁波的偏振状态是:
\begin{enumerate}[itemsep=0mm,label=\Alph*.]
    \item 线性偏振
    \item 圆偏振
    \item 椭圆偏振
    \item 自然光
\end{enumerate}

\item 真空中单色平面电磁波的时间平均能流密度 $\langle \boldsymbol{S} \rangle$ 与电场振幅 $E_0$ 的关系是:
\begin{enumerate}[itemsep=0mm,label=\Alph*.]
    \item $\langle S \rangle = \frac{1}{2 \mu_0 c} E_0^2$
    \item $\langle S \rangle = \epsilon_0 c E_0^2$
    \item $\langle S \rangle = \frac{c}{\epsilon_0} E_0^2$
    \item $\langle S \rangle = \frac{1}{2}\mu_0 c E_0^2$
\end{enumerate}

\item 电磁波的辐射压 $P_{\text{rad}}$(作用在完全吸收表面上的压强)与时间平均能流密度 $\langle S \rangle$ 的关系是:
\begin{enumerate}[itemsep=0mm,label=\Alph*.]
    \item $P_{\text{rad}} = \langle S \rangle$
    \item $P_{\text{rad}} = c \langle S \rangle$
    \item $P_{\text{rad}} = \frac{1}{c} \langle S \rangle$
    \item $P_{\text{rad}} = 2 \frac{1}{c} \langle S \rangle$
\end{enumerate}

\item 在一个线性、非色散、无源的介质 ($\epsilon, \mu, \sigma=0$) 中,电磁波的相速度 $v$ 与该介质的折射率 $n$ 的关系是:
\begin{enumerate}[itemsep=0mm,label=\Alph*.]
    \item $v = c n$
    \item $v = c \sqrt{\mu_r \epsilon_r}$
    \item $v = \frac{c}{n}$
    \item $v = \frac{1}{\sqrt{\epsilon \mu}}$
\end{enumerate}

\item 当电磁波从介质 1 垂直入射到介质 2 时,反射系数 $R$ (反射波强度与入射波强度之比) 的表达式为:
\begin{enumerate}[itemsep=0mm,label=\Alph*.]
    \item $R = \left(\frac{n_2 - n_1}{n_2 + n_1}\right)^2$
    \item $R = \left(\frac{n_1 - n_2}{n_1 + n_2}\right)^2$
    \item $R = \frac{n_2 - n_1}{n_2 + n_1}$
    \item $R = \frac{E_{0R}}{E_{0I}}$
\end{enumerate}

\item 电磁波从真空入射到良导体 ($\sigma \to \infty$) 表面时,电场振幅的透射系数 $t$ 的近似值是:
\begin{enumerate}[itemsep=0mm,label=\Alph*.]
    \item $t \approx 1$
    \item $t \approx 0$
    \item $t \approx \frac{2}{1+i}$
    \item $t \approx \frac{2\omega \epsilon_0}{i\sigma}$
\end{enumerate}

\item 在良导体中传播的电磁波,其电场振幅衰减到 $1/e$ 所经过的距离称为趋肤深度(Skin Depth) $\delta$。 $\delta$ 与频率 $\omega$ 的关系是:
\begin{enumerate}[itemsep=0mm,label=\Alph*.]
    \item $\delta \propto \omega$
    \item $\delta \propto \sqrt{\omega}$
    \item $\delta \propto 1/\omega$
    \item $\delta \propto 1/\sqrt{\omega}$
\end{enumerate}

\item 在存在色散(Dispersion)的介质中,波包的传播速度由哪个速度决定?
\begin{enumerate}[itemsep=0mm,label=\Alph*.]
    \item 相速度 $v_p$
    \item 群速度 $v_g$
    \item 光速 $c$
    \item $v_p$ 和 $v_g$ 的平均值
\end{enumerate}

\item 对于单色平面波 $\boldsymbol{E}(\boldsymbol{r}, t) = \boldsymbol{E}_0 e^{i(\boldsymbol{k} \cdot \boldsymbol{r} - \omega t)}$,$\boldsymbol{E}_0$ 必须满足 $\boldsymbol{k} \cdot \boldsymbol{E}_0 = 0$,这个条件来自于哪个麦克斯韦方程?
\begin{enumerate}[itemsep=0mm,label=\Alph*.]
    \item 法拉第感应定律 ($\nabla \times \boldsymbol{E} = -\frac{\partial \boldsymbol{B}}{\partial t}$)
    \item 安培-麦克斯韦定律 ($\nabla \times \boldsymbol{B} = \mu_0 \epsilon_0 \frac{\partial \boldsymbol{E}}{\partial t}$)
    \item 高斯定律 ($\nabla \cdot \boldsymbol{E} = 0$)
    \item 磁场无源定律 ($\nabla \cdot \boldsymbol{B} = 0$)
\end{enumerate}

\item 当电磁波从光疏介质(低 $n$)以大于布儒斯特角 ($\theta_B$) 的角度入射到光密介质(高 $n$)时:
\begin{enumerate}[itemsep=0mm,label=\Alph*.]
    \item 只有 $E$ 场垂直于入射面的分量发生全反射。
    \item 只有 $E$ 场平行于入射面的分量发生全反射。
    \item 两个分量都不会发生全反射。
    \item 两个分量都会发生全反射(若角度大于临界角 $\theta_c$)。
\end{enumerate}

\item 电磁波在良导体中传播时,电场和磁场之间的相位关系是:
\begin{enumerate}[itemsep=0mm,label=\Alph*.]
    \item $\boldsymbol{E}$ 和 $\boldsymbol{B}$ 同相。
    \item $\boldsymbol{E}$ 领先 $\boldsymbol{B}$ 相位角 $45^\circ$。
    \item $\boldsymbol{E}$ 落后 $\boldsymbol{B}$ 相位角 $45^\circ$。
    \item $\boldsymbol{E}$ 领先 $\boldsymbol{B}$ 相位角 $90^\circ$。
\end{enumerate}

\item 在一个具有截止频率 $\omega_c$ 的理想矩形波导中,若信号频率 $\omega < \omega_c$,则波导:
\begin{enumerate}[itemsep=0mm,label=\Alph*.]
    \item 内部发生全反射,波能无衰减传播。
    \item 无法传播,波能呈指数衰减。
    \item 以相速度 $v_p = c$ 传播。
    \item 以群速度 $v_g = c$ 传播。
\end{enumerate}

\item 对于真空中沿 $z$ 方向传播的波 $\boldsymbol{E}(\boldsymbol{r}, t) = E_0 \cos(kz - \omega t) \hat{\boldsymbol{x}}$,其瞬时波印廷矢量 $\boldsymbol{S}$ 在 $z$ 轴上的时间平均值为:
\begin{enumerate}[itemsep=0mm,label=\Alph*.]
    \item $\langle S \rangle = 0$
    \item $\langle S \rangle = \frac{E_0^2}{2\mu_0 c}$
    \item $\langle S \rangle = \frac{E_0^2}{\mu_0 c}$
    \item $\langle S \rangle = \frac{E_0^2}{4\mu_0 c}$
\end{enumerate}

\item 考虑电磁波在理想介质 ($\epsilon, \mu$) 中传播,其相速度 $v_p$ 和群速度 $v_g$ 的表达式为:
\begin{enumerate}[itemsep=0mm,label=\Alph*.]
    \item $v_p = c/\sqrt{\mu_r \epsilon_r}$, $v_g = v_p$
    \item $v_p = c/\sqrt{\mu_r \epsilon_r}$, $v_g = \frac{d\omega}{dk}$
    \item $v_p = \frac{d\omega}{dk}$, $v_g = \frac{\omega}{k}$
    \item $v_p = \sqrt{\frac{1}{\epsilon \mu}}$, $v_g = \frac{d\omega}{dk}$
\end{enumerate}

\item 对于一个从介质 1 ($\mu_1, \epsilon_1$) 垂直入射到介质 2 ($\mu_2, \epsilon_2$) 的电磁波,其电场透射系数 $t$ 的表达式是:
\begin{enumerate}[itemsep=0mm,label=\Alph*.]
    \item $t = \frac{2 \sqrt{\mu_1/\epsilon_1}}{\sqrt{\mu_1/\epsilon_1} + \sqrt{\mu_2/\epsilon_2}}$
    \item $t = \frac{2 \sqrt{\mu_2/\epsilon_2}}{\sqrt{\mu_1/\epsilon_1} + \sqrt{\mu_2/\epsilon_2}}$
    \item $t = \frac{2 \eta_2}{\eta_1 + \eta_2}$
    \item $t = \frac{2 \eta_1}{\eta_1 + \eta_2}$
\end{enumerate}

\item 在电磁波传播中,若电场和磁场的能量密度相等,即 $u_E = u_B$,则该介质必须满足哪个条件?
\begin{enumerate}[itemsep=0mm,label=\Alph*.]
    \item $\mu = \mu_0$
    \item $\epsilon = \epsilon_0$
    \item $\mu \epsilon = 1/c^2$
    \item 总是成立
\end{enumerate}

\item 当光从光密介质以大于临界角 $\theta_c$ 入射到光疏介质时,发生全反射。此时,在光疏介质中:
\begin{enumerate}[itemsep=0mm,label=\Alph*.]
    \item 没有电磁场存在,能量完全反射。
    \item 存在沿边界传播的倏逝波 (Evanescent Wave),但能量流为零。
    \item 存在沿边界传播的倏逝波,且能量沿边界传播。
    \item 存在正常的传播波,但振幅衰减很快。
\end{enumerate}

\item 某电磁波的电场表达式为 $\boldsymbol{E}(\boldsymbol{r}, t) = E_0 e^{-\alpha z} \cos(kz - \omega t) \hat{\boldsymbol{x}}$。这代表了:
\begin{enumerate}[itemsep=0mm,label=\Alph*.]
    \item 沿 $x$ 方向传播的衰减波。
    \item 沿 $z$ 方向传播的衰减波。
    \item 沿 $z$ 方向传播的非衰减波。
    \item 沿 $z$ 方向传播的驻波。
\end{enumerate}

\item 磁场 $\boldsymbol{B}$ 和电场 $\boldsymbol{E}$ 的表达式 ($\boldsymbol{B} = \nabla \times \boldsymbol{A}, \boldsymbol{E} = -\nabla V - \frac{\partial \boldsymbol{A}}{\partial t}$) 对一个规范变换 $V' = V - \frac{\partial \lambda}{\partial t}, \boldsymbol{A}' = \boldsymbol{A} + \nabla \lambda$ 保持不变,是因为:
\begin{enumerate}[itemsep=0mm,label=\Alph*.]
    \item 麦克斯韦方程组是线性方程组。
    \item 势函数 $V$ 和 $\boldsymbol{A}$ 必须满足连续性方程。
    \item $\nabla \times (\nabla \lambda) = 0$ 和 $\frac{\partial}{\partial t}(\nabla \lambda) = \nabla (\frac{\partial \lambda}{\partial t})$。
    \item 洛伦兹力公式 $\boldsymbol{F} = q(\boldsymbol{E} + \boldsymbol{v} \times \boldsymbol{B})$ 具有规范不变性。
\end{enumerate}

\item 在洛伦兹规范(Lorentz Gauge)中,标量势 $V$ 满足的非齐次波动方程是:
\begin{enumerate}[itemsep=0mm,label=\Alph*.]
    \item $\nabla^2 V - \mu_0 \epsilon_0 \frac{\partial^2 V}{\partial t^2} = 0$
    \item $\nabla^2 V = -\frac{1}{\epsilon_0}\rho$
    \item $\nabla^2 V - \mu_0 \epsilon_0 \frac{\partial^2 V}{\partial t^2} = -\frac{1}{\epsilon_0}\rho$
    \item $\nabla^2 V - \mu_0 \epsilon_0 \frac{\partial^2 V}{\partial t^2} = -\mu_0 \boldsymbol{J}$
\end{enumerate}

\item 库仑规范(Coulomb Gauge) $\nabla \cdot \boldsymbol{A}=0$ 在瞬时形式上使标量势 $V$ 满足泊松方程 ($\nabla^2 V = -\rho/\epsilon_0$)。这暗示着库仑规范下的 $V$:
\begin{enumerate}[itemsep=0mm,label=\Alph*.]
    \item 必须是零。
    \item 与时间无关。
    \item 瞬时响应电荷分布 $\rho(\boldsymbol{r}, t)$。
    \item 满足 $\nabla \cdot \boldsymbol{A} + \mu_0 \epsilon_0 \frac{\partial V}{\partial t} = 0$。
\end{enumerate}

\item 延迟标量势 $V(\boldsymbol{r}, t)$ 的定义式为 $V(\boldsymbol{r}, t) = \frac{1}{4\pi \epsilon_0} \int \frac{\rho(\boldsymbol{r}', t_r)}{|\boldsymbol{r}-\boldsymbol{r}'|} d\tau'$。其中延迟时间 $t_r$ 满足:
\begin{enumerate}[itemsep=0mm,label=\Alph*.]
    \item $t_r = t + \frac{|\boldsymbol{r}-\boldsymbol{r}'|}{c}$
    \item $t_r = t - \frac{|\boldsymbol{r}-\boldsymbol{r}'|}{c}$
    \item $t_r = t$
    \item $t_r = t - \frac{|\boldsymbol{r}'|}{c}$
\end{enumerate}

\item Jefimenko方程明确了在时变场中,电场 $\boldsymbol{E}(\boldsymbol{r}, t)$ 产生于哪些物理量?
\begin{enumerate}[itemsep=0mm,label=\Alph*.]
    \item 仅延迟电荷 $\rho(\boldsymbol{r}', t_r)$。
    \item 仅延迟电流 $\boldsymbol{J}(\boldsymbol{r}', t_r)$。
    \item 延迟电荷 $\rho$ 及其对时间的导数 $\frac{\partial \rho}{\partial t_r}$,以及延迟电流 $\boldsymbol{J}$ 及其对时间的导数 $\frac{\partial \boldsymbol{J}}{\partial t_r}$。
    \item 仅延迟电荷 $\rho$ 和延迟电流 $\boldsymbol{J}$。
\end{enumerate}

\item 考虑一个匀速直线运动的点电荷 $q$,其 $\boldsymbol{E}$ 场在运动方向上的空间分布特点是:
\begin{enumerate}[itemsep=0mm,label=\Alph*.]
    \item 场线密度在运动方向上被压缩(前方增强,后方减弱)。
    \item 场线密度在垂直运动方向上被压缩(垂直方向增强)。
    \item 场线分布仍然是径向对称的,但振幅随速度减小。
    \item 场线分布与静止时完全相同。
\end{enumerate}

\item Liénard-Wiechert 势中,分母处的因子 $\mathcal{R} = R - \boldsymbol{R} \cdot \boldsymbol{v}/c$ 的物理意义不包括:
\begin{enumerate}[itemsep=0mm,label=\Alph*.]
    \item 考虑了电磁信号有限的传播速度 $c$。
    \item 考虑了源点在信号发出后仍在运动。
    \item 使得势函数 $V$ 随距离的衰减不再是简单的 $1/R$。
    \item 使得势函数 $V$ 具有洛伦兹协变性。
\end{enumerate}

\item 对于运动点电荷的电场 $\boldsymbol{E}$ 表达式,可将其拆分为速度场(Velocity Field, $\boldsymbol{E}_v$)和加速度场(Acceleration Field, $\boldsymbol{E}_a$)。下列哪个陈述是正确的?
\begin{enumerate}[itemsep=0mm,label=\Alph*.]
    \item $\boldsymbol{E}_v$ 场的强度随距离 $R$ 衰减为 $1/R^3$。
    \item $\boldsymbol{E}_a$ 场的强度随距离 $R$ 衰减为 $1/R^2$。
    \item $\boldsymbol{E}_a$ 场的总能量流(辐射功率)与距离 $R$ 无关。
    \item $\boldsymbol{E}_v$ 场总是垂直于速度 $\boldsymbol{v}$ 的。
\end{enumerate}

\item 为了使 $\boldsymbol{E}$ 场满足法拉第定律 $\nabla \times \boldsymbol{E} = -\frac{\partial \boldsymbol{B}}{\partial t}$,势 $\boldsymbol{A}$ 和 $V$ 必须满足:
\begin{enumerate}[itemsep=0mm,label=\Alph*.]
    \item $\nabla \cdot \boldsymbol{A} = 0$
    \item $\boldsymbol{E} = -\nabla V$
    \item $\boldsymbol{B} = \nabla \times \boldsymbol{A}$
    \item $\boldsymbol{E} = -\nabla V - \frac{\partial \boldsymbol{A}}{\partial t}$
\end{enumerate}

\item 任意电荷-电流分布 $\rho(\boldsymbol{r}, t)$ 和 $\boldsymbol{J}(\boldsymbol{r}, t)$ 产生的 $\boldsymbol{E}$ 场,在远场区 ($R \to \infty$),哪一项贡献(或哪一项与 $R$ 的关系)决定了电磁辐射?
\begin{enumerate}[itemsep=0mm,label=\Alph*.]
    \item 仅由延迟电荷项 $\propto 1/R^2$ 贡献。
    \item 仅由延迟电流项 $\propto 1/R$ 贡献。
    \item 由所有项中随 $1/R$ 衰减的分量贡献。
    \item 仅由延迟电荷项 $\propto 1/R$ 贡献。
\end{enumerate}

\item 判别电磁场中是否存在不可逆能量传输(辐射)的决定性条件是:
\begin{enumerate}[itemsep=0mm,label=\Alph*.]
    \item 电场 $\boldsymbol{E}$ 随距离 $r$ 衰减为 $1/r^2$。
    \item 磁场 $\boldsymbol{B}$ 随距离 $r$ 衰减为 $1/r^3$。
    \item 波印廷矢量 $\boldsymbol{S}$ 在远场区随距离 $r$ 衰减不快于 $1/r^2$。
    \item 源点的电荷和电流必须随时间呈正弦振荡。
\end{enumerate}

\item 对于一个振荡电偶极子 $\boldsymbol{p}(t) = p_0 \cos(\omega t) \hat{\boldsymbol{z}}$,其辐射场 $\boldsymbol{E}_{rad}$ 在远场区 ($r \to \infty$) 的表达式中,下列关于其 $r$ 依赖性和角度 $\theta$ 依赖性的描述正确的是($\theta$ 为与 $\hat{\boldsymbol{z}}$ 轴的夹角):
\begin{enumerate}[itemsep=0mm,label=\Alph*.]
    \item 振幅 $\propto 1/r^2$;角分布 $\propto \sin^2 \theta$。
    \item 振幅 $\propto 1/r$;角分布 $\propto \cos^2 \theta$。
    \item 振幅 $\propto 1/r$;角分布 $\propto \sin^2 \theta$。
    \item 振幅 $\propto 1/r^2$;角分布 $\propto \cos \theta$。
\end{enumerate}

\item 在多极矩展开中,电偶极子辐射是最低阶的辐射。其辐射功率 $P$ 与振荡频率 $\omega$ 的关系是:
\begin{enumerate}[itemsep=0mm,label=\Alph*.]
    \item $P \propto \omega^2$
    \item $P \propto \omega^4$
    \item $P \propto \omega^6$
    \item $P \propto 1/\omega^2$
\end{enumerate}

\item 考虑一个半径为 $a$ 的电流环 $I(t)$,其磁偶极矩为 $\boldsymbol{m}(t)$。磁偶极辐射相对于电偶极辐射的相对强度由哪个无量纲小参数决定?
\begin{enumerate}[itemsep=0mm,label=\Alph*.]
    \item $v/c$
    \item $\omega a / c$
    \item $a/r$
    \item $\mu_0 \epsilon_0$
\end{enumerate}

\item Larmor 公式 $P = \frac{\mu_0 q^2}{6\pi c} |\boldsymbol{a}|^2$ 给出了非相对论运动下点电荷的瞬时辐射功率。该公式成立的前提是:
\begin{enumerate}[itemsep=0mm,label=\Alph*.]
    \item 电荷必须做匀速圆周运动。
    \item 观测点必须在辐射区。
    \item 电荷的速度远小于光速且加速度 $\boldsymbol{a}$ 远小于 $c^2/R$。
    \item 电荷的速度远小于光速且辐射反作用力可忽略。
\end{enumerate}

\item 对于匀速直线运动的点电荷,其辐射功率 $P$ 为:
\begin{enumerate}[itemsep=0mm,label=\Alph*.]
    \item $P = \frac{\mu_0 q^2}{6\pi c} |\boldsymbol{a}|^2$
    \item $P = 0$
    \item $P$ 与速度 $\boldsymbol{v}$ 成正比。
    \item $P$ 与速度 $\boldsymbol{v}$ 和加速度 $\boldsymbol{a}$ 的乘积成正比。
\end{enumerate}

\item 在 Liénard 公式中,描述点电荷辐射功率角分布 $dP/d\Omega$ 的辐射场($\boldsymbol{E}_{rad}$)与总电场 $\boldsymbol{E}$ 的衰减行为在远场区分别满足:
\begin{enumerate}[itemsep=0mm,label=\Alph*.]
    \item $\boldsymbol{E}_{rad} \propto 1/R^2$; $\boldsymbol{E} \propto 1/R$。
    \item $\boldsymbol{E}_{rad} \propto 1/R$; $\boldsymbol{E} \propto 1/R^2$。
    \item $\boldsymbol{E}_{rad} \propto 1/R$; $\boldsymbol{E} \propto 1/R$。
    \item $\boldsymbol{E}_{rad} \propto 1/R$; $\boldsymbol{E} \propto 1/R$ 和 $1/R^2$ 的组合。
\end{enumerate}

\item 经典电动力学中,Abraham-Lorentz 辐射反作用力 $\boldsymbol{F}_{rad}$ 的表达式 $\boldsymbol{F}_{rad} = \frac{\mu_0 q^2}{6\pi c} \dot{\boldsymbol{a}}$ 带来了哪个物理学上的难题?
\begin{enumerate}[itemsep=0mm,label=\Alph*.]
    \item 违反了动量守恒。
    \item 违反了能量守恒。
    \item 导致了预加速(Pre-acceleration)和失控解(Runaway Solutions)。
    \item 辐射功率与加速度的平方成正比。
\end{enumerate}

\item 当点电荷处于极端相对论运动 ($v \to c$) 时,其辐射功率的角分布 $dP/d\Omega$ 表现出何种特性?
\begin{enumerate}[itemsep=0mm,label=\Alph*.]
    \item 辐射方向与加速度方向垂直。
    \item 辐射集中在与 $\boldsymbol{v}$ 方向相反的极窄锥形区域内。
    \item 辐射集中在与 $\boldsymbol{v}$ 方向平行的极窄锥形区域内。
    \item 辐射分布与非相对论情况相似,呈 $\sin^2\theta$ 分布。
\end{enumerate}

\item 四极矩辐射 (Quadrupole Radiation) 相比于电偶极矩辐射,其总辐射功率 $P$ 的 $r$ 依赖性和 $\omega$ 依赖性分别是:
\begin{enumerate}[itemsep=0mm,label=\Alph*.]
    \item $P \propto 1/r^2, P \propto \omega^4$
    \item $P \propto r^0, P \propto \omega^6$
    \item $P \propto 1/r, P \propto \omega^6$
    \item $P \propto r^0, P \propto \omega^8$
\end{enumerate}

\item 在狭义相对论中,电动力学与力学的一致性是通过哪个原理建立起来的?
\begin{enumerate}[itemsep=0mm,label=\Alph*.]
    \item 洛伦兹力公式的协变性。
    \item 惯性系中所有物理定律形式不变。
    \item 光速不变原理。
    \item 场和源都是四维张量或四维矢量。
\end{enumerate}

\item 一个四维矢量 $\boldsymbol{A}^\mu = (A^0, \boldsymbol{A})$,其零分量 $A^0$ 在洛伦兹变换下的变换特性通常是什么?
\begin{enumerate}[itemsep=0mm,label=\Alph*.]
    \item 保持不变。
    \item 与空间分量 $\boldsymbol{A}$ 的变化无关。
    \item 与空间分量 $\boldsymbol{A}$ 耦合。
    \item 恒等于零。
\end{enumerate}

\item 四维电流密度 $J^\mu$ 的定义是:
\begin{enumerate}[itemsep=0mm,label=\Alph*.]
    \item $(\rho, \boldsymbol{J})$
    \item $(\rho c, \boldsymbol{J})$
    \item $(\rho c, \boldsymbol{J}/c)$
    \item $(\rho c, \boldsymbol{J} \gamma)$
\end{enumerate}

\item 在电动力学中,四维势 $A^\mu = (V/c, \boldsymbol{A})$ 满足的洛伦兹规范条件,用四维表示应为:
\begin{enumerate}[itemsep=0mm,label=\Alph*.]
    \item $\nabla \cdot \boldsymbol{A} = 0$
    \item $\partial_\mu A^\mu = 0$
    \item $A_\mu A^\mu = \text{常数}$
    \item $\nabla^2 V - \frac{1}{c^2} \frac{\partial^2 V}{\partial t^2} = -\frac{\rho}{\epsilon_0}$
\end{enumerate}

\item 电磁场张量 $F^{\mu\nu}$ 是一个反对称的二阶张量。其分量 $F^{10}$ 对应哪个电磁场分量?
\begin{enumerate}[itemsep=0mm,label=\Alph*.]
    \item $E_x / c$
    \item $-E_x / c$
    \item $B_x$
    \item $B_z$
\end{enumerate}

\item 麦克斯韦方程组用电磁场张量 $F^{\mu\nu}$ 和四维电流密度 $J^\mu$ 表示,其中描述场与源关系的方程(非齐次方程)是:
\begin{enumerate}[itemsep=0mm,label=\Alph*.]
    \item $\partial_\lambda F_{\mu\nu} + \partial_\mu F_{\nu\lambda} + \partial_\nu F_{\lambda\mu} = 0$
    \item $\partial_\nu F^{\mu\nu} = \mu_0 J^\mu$
    \item $\partial_\mu F^{\mu\nu} = 0$
    \item $F^{\mu\nu} F_{\mu\nu} = \text{常数}$
\end{enumerate}

\item 考虑一个相对论性粒子所受的四维洛伦兹力 $K^\mu$。其中零分量 $K^0$ 的物理意义是什么?
\begin{enumerate}[itemsep=0mm,label=\Alph*.]
    \item 粒子所受的瞬时力。
    \item 粒子在单位时间内的动量变化率。
    \item 粒子所受的辐射反作用力。
    \item 洛伦兹力对粒子做功的功率 $P$ 与 $c$ 的比值 $P/c$。
\end{enumerate}

\item 在惯性系 $S'$ 中观察到一个静止的电偶极子。若 $S'$ 相对于 $S$ 系以速度 $\boldsymbol{v}=v\hat{\boldsymbol{x}}$ 运动,则在 $S$ 系中观测到的场是:
\begin{enumerate}[itemsep=0mm,label=\Alph*.]
    \item 仅有电场,没有磁场。
    \item 仅有磁场,没有电场。
    \item 既有电场,也有磁场。
    \item 仅有一个与 $\boldsymbol{v}$ 平行的电场。
\end{enumerate}

\item 描述电磁场能量和动量密度的张量是:
\begin{enumerate}[itemsep=0mm,label=\Alph*.]
    \item 电磁场张量 $F^{\mu\nu}$
    \item 能量-动量四维矢量 $P^\mu$
    \item 波印廷矢量 $\boldsymbol{S}$
    \item 能量-动量张量 $T^{\mu\nu}$
\end{enumerate}

\item 麦克斯韦方程组的协变形式,特别是 $\partial_\nu F^{\mu\nu} = \mu_0 J^\mu$,保证了电荷守恒定律(连续性方程)的自然满足。这是通过哪个数学性质实现的?
\begin{enumerate}[itemsep=0mm,label=\Alph*.]
    \item 洛伦兹规范条件。
    \item 电磁场张量 $F^{\mu\nu}$ 的反对称性。
    \item 麦克斯韦方程组的线性性质。
    \item 闵可夫斯基度规 $g_{\mu\nu}$ 的对角性。
\end{enumerate}

 

\end{enumerate}



\newpage
\section{简答题}

\begin{enumerate}

\item 狄拉克 $\delta$ 函数的性质与物理意义
\begin{enumerate}
    \item 写出三维狄拉克 $\delta$ 函数 $\delta^3(\boldsymbol{r})$ 的定义及其两个关键性质(积分性质和筛选性质)。
    \item 解释为什么 $\nabla^2 (1/r) = -4\pi \delta^3(\boldsymbol{r})$ 在电磁学中至关重要。请说明它如何将泊松方程和库仑定律联系起来。
    \item 在计算 $\nabla^2 (1/r)$ 时,为什么不能简单地直接应用微分算子而忽略原点 $r=0$ 处的特殊性?
\end{enumerate}

\item 矢量积分的四大基本定理
\begin{enumerate}
    \item 简述矢量微积分中的四个基本定理(梯度定理、高斯散度定理、斯托克斯旋度定理、以及基本积分定理)的一般形式。
    \item 阐述这四个定理在数学形式上的共同特征。
    \item 解释这些定理在物理学中最重要的意义,即它们如何描述边界上的量与体积/区域内部的量之间的关系。
\end{enumerate}

\item 导体静电平衡性质的证明与应用
\begin{enumerate}
    \item 利用静电学的基本方程,证明处于静电平衡状态的完美导体内部,净电荷密度 $\rho$ 必须为零。
    \item 证明处于静电平衡的导体表面,电场强度 $\boldsymbol{E}$ 必须垂直于导体表面。
    \item 解释静电屏蔽的原理,并说明其在实际应用中的有效性如何体现在导体的静电平衡性质中。
\end{enumerate}

\item 高斯定律和库仑定律的关系与静电场的旋度
\begin{enumerate}
    \item 说明高斯定律和库仑定律之间的关系。为什么在静电学中,仅凭高斯定律不足以完全确定电场 $\boldsymbol{E}$?
    \item 在静电学中,$\nabla \times \boldsymbol{E} = 0$ 这一微分方程的物理意义是什么?请从功和场的保守性角度阐述。
    \item 证明 $V(\boldsymbol{r})=\frac{1}{4\pi\epsilon_{0}}\int\frac{\rho(\boldsymbol{r}')}{|\boldsymbol{r}-\boldsymbol{r}'|}d\tau'$ 形式的电势,自动满足 $\nabla \times \boldsymbol{E} = 0$。
\end{enumerate}

\item  唯一性定理与边值条件
\begin{enumerate}
    \item 阐述静电学第一唯一性定理和第二唯一性定理对求解静电场边值问题的意义。
    \item 简述 Dirichlet 边界条件和 Neumann 边界条件在物理上的含义,并说明当边界条件给出后,为什么解是唯一的(简要说明证明思路)。
    \item 在实际应用中,如果边界 $S$ 上一部分满足 Dirichlet 条件,另一部分满足 Neumann 条件,此时唯一性定理是否仍然成立?为什么?
\end{enumerate}

\item 静电能的物理本质与点电荷问题
\begin{enumerate}
    \item 解释静电能公式 $W = \frac{1}{2}\int \rho V d\tau$(电荷观点)和 $W = \frac{\epsilon_0}{2}\int E^2 d\tau$(场观点)的物理含义,以及它们何时是等价的。
    \item 证明对于一个点电荷 $q$,由 $\frac{\epsilon_0}{2}\int E^2 d\tau$ 计算出的能量是无穷大。这种自能无穷大的结果揭示了经典电磁学在描述点电荷时的何种根本性困难?
\end{enumerate}


\item 电介质中 $\boldsymbol{E}$ 场和 $\boldsymbol{D}$ 场的物理意义与应用
\begin{enumerate}
    \item 阐述电场强度 $\boldsymbol{E}$ 场和辅助电场 $\boldsymbol{D}$ 场的物理意义和根本区别。
    \item 解释引入 $\boldsymbol{D}$ 场的物理动机。为什么在处理电介质边界条件问题时,$\boldsymbol{D}$ 场比 $\boldsymbol{E}$ 场更方便?
    \item 在一个电介质内部,挖出一个扁平盘状腔和一个细长针状腔时,腔内 $\boldsymbol{E}$ 场和 $\boldsymbol{D}$ 场与介质中 $\boldsymbol{E}$ 场和 $\boldsymbol{D}$ 场的关系,并解释其物理原因。
\end{enumerate}

\item 束缚电荷的本质与电介质的静电能
\begin{enumerate}
    \item 描述电介质中束缚电荷的微观起源,并解释为什么束缚电荷的净效应总是削弱了自由电荷产生的电场。
    \item 推导静电能公式 $W = \frac{1}{2} \int \boldsymbol{D} \cdot \boldsymbol{E} d\tau$(适用于线性电介质),并解释这个能量表达式与真空中 $W_0 = \frac{1}{2} \epsilon_0 \int E^2 d\tau$ 的物理区别。
\end{enumerate}


\item 静磁学的基本原理与电磁场的对称性
\begin{enumerate}
    \item 解释静磁学中 $\nabla \cdot \boldsymbol{B} = 0$ 和静电学中 $\nabla \times \boldsymbol{E} = 0$ 这两个方程的物理意义和根本差异。
    \item 阐述“磁单极子”的概念。如果磁单极子存在,静磁学的两个基本方程将如何修改?这会给磁矢势 $\boldsymbol{A}$ 的定义带来什么影响?
\end{enumerate}

\item 磁矢势的规范自由度 
\begin{enumerate}
    \item 解释磁矢势 $\boldsymbol{A}$ 的“规范自由度” 是什么。写出通用的规范变换公式。
    \item 为什么磁矢势 $\boldsymbol{A}$ 的引入(而非直接使用 $\boldsymbol{B}$)在理论物理中是必要的?除了简化方程,它的更深层次的物理意义是什么?(提示:考虑量子力学和延迟势。)
\end{enumerate}

\item $\boldsymbol{B}$ 场和 $\boldsymbol{H}$ 场的物理意义及应用区别
\begin{enumerate}
    \item 阐述磁感应强度 $\boldsymbol{B}$ 场和辅助磁场 $\boldsymbol{H}$ 场的物理意义和根本区别。
    \item 为什么在计算一个被均匀磁化的物体(如磁棒)产生的磁场时,可以采用两种看似不同的方法(束缚电流法和 $\boldsymbol{H}$ 场法),但最终得到相同的 $\boldsymbol{B}$ 场?
    \item 简述在物质内部挖出一个扁平盘状腔和一个细长针状腔时,腔内 $\boldsymbol{B}$ 场和 $\boldsymbol{H}$ 场与介质中 $\boldsymbol{B}$ 场和 $\boldsymbol{H}$ 场的关系,并解释其物理原因。
\end{enumerate}

\item 磁场的本质与永磁体的能量
\begin{enumerate}
    \item 描述经典电动力学中永磁体的磁化强度 $\boldsymbol{M}$ 场的微观起源,并解释为什么永磁体的磁场被称为“束缚场”。
    \item 在构建永磁体时(例如铁),为何需要消耗能量?这部分能量存储在哪里?写出永磁体在外部磁场 $\boldsymbol{B}_{ext}$ 中的磁能(或势能)公式,并解释其物理含义。
\end{enumerate}

\item 解释 $\nabla \cdot (\nabla \times \boldsymbol{F}) = 0$ 和 $\nabla \times (\nabla f) = 0$ 这两个矢量恒等式在麦克斯韦方程组中的体现和物理意义。

\item 磁荷和电磁场的对偶性
如果假设存在磁单极子(磁荷密度 $\rho_m$ 和磁流密度 $\boldsymbol{J}_m$),请写出引入磁荷后的麦克斯韦方程组。并简述电磁场($\boldsymbol{E}$ 和 $\boldsymbol{B}$)在电荷和磁荷存在时所体现的对偶性。

\item 法拉第定律的积分和微分形式的适用性
法拉第电磁感应定律的积分形式 $\oint \boldsymbol{E} \cdot d\boldsymbol{l} = - d\Phi_B/dt$ 和微分形式 $\nabla \times \boldsymbol{E} = - \partial \boldsymbol{B}/\partial t$ 之间有何根本区别?哪种形式对于描述动生电动势更直接?

\item  简述稳恒电流在数学上必须满足的两个条件,并说明在均匀导电介质中,稳恒电流的电荷分布特征。

\item 详细阐述波印廷定理(积分形式) $\frac{dW}{dt}=-\frac{d}{dt}\int_{\mathcal{V}}u~d\tau-\oint_{\mathcal{S}}\boldsymbol{S}\cdot da$ 中,电磁场能量守恒的含义。特别要指出电磁场能量本身是否守恒?如果否,那么总能量是如何守恒的?
\item 阐述在电动力学中,动量守恒是如何被麦克斯韦应力张量和场动量概念所“拯救”的。请解释在 $\frac{d\boldsymbol{P}_{mech}}{dt} = \oint_{\mathcal{S}} \overleftrightarrow{T} \cdot d\boldsymbol{a} - \frac{d\boldsymbol{P}_{em}}{dt}$ 这一动量守恒方程中,应力张量 $\overleftrightarrow{T}$ 和场动量 $\boldsymbol{P}_{em}$ 的物理作用。
\item 在无源线性介质中 ($\rho=0, \boldsymbol{J}=0, \epsilon, \mu$),电磁波是横波。
\begin{enumerate}
    \item 证明对于单色平面波 $\boldsymbol{E}(\boldsymbol{r}, t) = \boldsymbol{E}_0 e^{i(\boldsymbol{k} \cdot \boldsymbol{r} - \omega t)}$,由高斯定律 $\nabla \cdot \boldsymbol{E} = 0$ 可推导出 $\boldsymbol{k} \perp \boldsymbol{E}$。
    \item 简述为什么在有源介质(例如导体)中,电磁波不再是严格的横波,而是存在纵向分量?
\end{enumerate}


\item 当电磁波在光密介质和光疏介质边界上发生全反射时,光疏介质中会产生倏逝波。
\begin{enumerate}
    \item 解释什么是倏逝波,并指出其沿着垂直于边界的方向上的特性。
    \item 简述倏逝波是否携带能量?如果是,其能量流的方向是怎样的?为什么全反射仍然是“全”反射?
\end{enumerate}

\item 群速度和相速度的区别与联系
\begin{enumerate}
    \item 简述相速度 $v_p$ 和群速度 $v_g$ 的物理意义。
    \item 在何种介质中 $v_p$ 和 $v_g$ 相等?在何种介质中 $v_g > v_p$(反常色散),以及 $v_g < v_p$(正常色散)?
    \item 物理信息(例如信号)的传播速度由哪个速度决定?为什么?
\end{enumerate}


\item 与在真空中的传播特性相比,电磁波在良导体 ($\sigma \to \infty$) 中传播时,其特性发生了哪些根本性的改变?请从以下三个方面阐述:
\begin{enumerate}
    \item 衰减特性(振幅和距离的关系)。
    \item 相速度和波长。
    \item $\boldsymbol{E}$ 场和 $\boldsymbol{B}$ 场的相位和振幅关系。
\end{enumerate}
 
 
\item 详细阐述“延迟势”在电动力学中的物理意义,特别是它如何体现电磁因果律。请说明为什么在静电学和静磁学中,势的表达式(如库仑势和毕奥-萨伐尔势)看起来是瞬时的,但这并不与因果律矛盾。

\item 比较和对比洛伦兹规范和库仑规范。特别要说明:
\begin{enumerate}
    \item 它们各自的定义条件是什么?
    \item 它们各自导致势函数 $V$ 和 $\boldsymbol{A}$ 满足的微分方程的特点(是否解耦、是否是波动方程)?
    \item 它们各自在理论(如相对论)和实际应用(如静磁场)中的优势。
\end{enumerate} 

\item Liénard-Wiechert 场与辐射的起源
\begin{enumerate}
    \item Liénard-Wiechert 场表达式中,电场 $\boldsymbol{E}$ 可以分解为速度场($\boldsymbol{E}_v$)和加速度场( $\boldsymbol{E}_a$)。请说明 $\boldsymbol{E}_v$ 和 $\boldsymbol{E}_a$ 在远场区 ($R \to \infty$) 的 $R$ 依赖性,并指出哪个分量负责电磁辐射?
    \item 解释 $\boldsymbol{E}_v$ 和 $\boldsymbol{E}_a$ 场的物理意义。为什么只有 $\boldsymbol{E}_a$ 场才能产生不可逆的能量流(即辐射)?
\end{enumerate}

\item 辐射反作用力的经典难题
\begin{enumerate}
    \item 辐射反作用力 $\boldsymbol{F}_{rad}$ (Abraham-Lorentz 公式) 是如何从能量守恒角度推导出来的?其推导的基本假设是什么?
    \item 详细阐述 Abraham-Lorentz 公式 $\boldsymbol{F}_{rad} = \frac{\mu_0 q^2}{6\pi c} \dot{\boldsymbol{a}}$ 在物理上导致的两个主要难题:预加速问题和失控解问题。这些难题如何表明经典电动力学在微观尺度上的局限性?
\end{enumerate}

\item $\boldsymbol{E}$ 场和 $\boldsymbol{B}$ 场的相对论本质
\begin{enumerate}
    \item 简述在电动力学与相对论的框架下,电场 $\boldsymbol{E}$ 和磁场 $\boldsymbol{B}$ 的物理本质是什么?它们如何不再是独立的场,而是统一于一个单一的数学对象?
    \item 以一个静止点电荷的例子进行说明:在 $S$ 系中观测到一个静止电荷 $q$ 产生纯静电场 $\boldsymbol{E}$。当切换到一个相对 $S$ 系以速度 $\boldsymbol{v}$ 运动的 $S'$ 系时,场 $\boldsymbol{E}'$ 和 $\boldsymbol{B}'$ 如何产生?这一现象如何体现了相对论对电磁学的统一?
\end{enumerate}

\item 麦克斯韦方程组的协变性
\begin{enumerate}
    \item 写出麦克斯韦方程组的四维协变形式,并指出每个方程对应的经典麦克斯韦方程。
    \item 解释“协变性”的物理含义。为什么麦克斯韦方程组的协变形式是狭义相对论成功的关键证据之一?
\end{enumerate}

\end{enumerate}


%%%%%%%%%%%%%%%%%%%%%%%%%%%%%%%%%%%%%%%%%%%
\newpage
\section{计算题}
%%%%%%%%%%%%%%%%%%%%%%%%%%%%%%%%%%%%%%%%%%%

%%%-----------------------%%%
%\subsection{量子力学背景知识}
%%%-----------------------%%%
%
%\begin{mdframed}
%\begin{quotation}
%\noindent 物理常数:
%\begin{enumerate}
%    \item 真空中的光速:$c \approx 3.00\times10^{8}\,\mathrm{m}/\mathrm{s}$
%    \item 普朗克常数:
%        $h \approx 6.63\times10^{-34}\,\mathrm{J}\cdot\mathrm{s} \approx 4.14\times10^{-15}\,\mathrm{eV} $
%    \item 电子质量:$m \approx 9.11\times10^{-31}\,\mathrm{kg} \approx 0.51\,\mathrm{MeV}/c^{2}$
%\end{enumerate}
%\end{quotation}
%\end{mdframed}

\begin{enumerate}
%  \item 设 $\boldsymbol{A} = 2x\hat{x} - yz\hat{y} + z^2\hat{z}$,求:
%  \begin{enumerate}
%    \item $\nabla\cdot\boldsymbol{A}$;
%    \item $\nabla\times\boldsymbol{A}$。
%  \end{enumerate}

\item 考虑矢量场 $\boldsymbol{v} = x^2 \hat{\boldsymbol{x}} + 2 y z \hat{\boldsymbol{y}} + y^2 \hat{\boldsymbol{z}}$。
\begin{enumerate}
    \item 计算 $\boldsymbol{v}$ 的散度 $\nabla \cdot \boldsymbol{v}$。
    \item 验证高斯散度定理 $\oint_S \boldsymbol{v} \cdot d\boldsymbol{a} = \int_V (\nabla \cdot \boldsymbol{v}) d\tau$,其中 $V$ 是由六个面 $x=0, 1; y=0, 1; z=0, 1$ 围成的单位立方体。
\end{enumerate}

\item 考虑矢量场 $\boldsymbol{v} = y \hat{\boldsymbol{x}} + 2 x \hat{\boldsymbol{y}}$。
\begin{enumerate}
    \item 计算 $\boldsymbol{v}$ 的旋度 $\nabla \times \boldsymbol{v}$。
    \item 验证斯托克斯定理 $\oint_C \boldsymbol{v} \cdot d\boldsymbol{l} = \int_S (\nabla \times \boldsymbol{v}) \cdot d\boldsymbol{a}$,其中 $S$ 是位于 $z=0$ 平面上,由 $x=0, 1$ 和 $y=0, 1$ 围成的单位正方形区域,边界 $C$ 沿逆时针方向。
\end{enumerate}

\item 在柱坐标系 $(s, \phi, z)$ 中,标量函数 $T$ 的拉普拉斯算子 $\nabla^2 T$ 为:
$$
\nabla^2 T = \frac{1}{s} \frac{\partial}{\partial s} \left( s \frac{\partial T}{\partial s} \right) + \frac{1}{s^2} \frac{\partial^2 T}{\partial \phi^2} + \frac{\partial^2 T}{\partial z^2}
$$
考虑一个沿 $z$ 轴无限长、半径为 $R$ 的圆柱体。圆柱体内 ($s < R$) 的电势 $V$ 仅依赖于径向距离 $s$ 和方位角 $\phi$,即 $V = V(s, \phi)$。
\begin{enumerate}
    \item 如果圆柱体内没有自由电荷,写出 $V(s, \phi)$ 满足的微分方程。
    \item 假设 $V(s, \phi)$ 具有特殊形式 $V(s, \phi) = s^n \cos(n\phi)$,其中 $n$ 是非负整数。将此形式代入 (a) 的微分方程中,证明 $V(s, \phi)$ 确实是一个可能的解。
    \item 证明 $\nabla \cdot (\nabla V) = 0$ 是恒成立的。
\end{enumerate}

\item 非均匀带电球体的电场与电势
一个半径为 $R$ 的球体内部,电荷密度 $\rho(\boldsymbol{r})$ 呈非均匀分布,由下式给出:
$$
\rho(r) = k r^2
$$
其中 $k$ 是常数,$r$ 是到球心 $O$ 的距离。假设球体周围是真空。
\begin{enumerate}
    \item 计算球体的总电荷 $Q_{total}$。
    \item 利用高斯定律,求出球体内部 ($r < R$) 的电场 $\boldsymbol{E}_{in}(\boldsymbol{r})$。
    \item 求出球体内部 ($r < R$) 的电势 $V_{in}(r)$,以无穷远处的电势为零作为参考。
\end{enumerate}

\item 均匀带电圆盘的电势与电场
一个半径为 $R$ 的圆盘,均匀带电,面电荷密度为 $\sigma$。
\begin{enumerate}
    \item 计算圆盘中心轴线($z$ 轴)上任意一点 $P(0, 0, z)$ 的电势 $V(z)$。
    \item 利用 $\boldsymbol{E} = -\nabla V$,求出点 $P$ 处的电场 $\boldsymbol{E}(z)$。
    \item 分析 $V(z)$ 和 $\boldsymbol{E}(z)$ 在 $z \gg R$ 时的近似形式,并解释物理意义。
\end{enumerate}

\item 考虑一个半径为 $R$ 的球体,带有均匀体电荷密度 $\rho_0$。
\begin{enumerate}
    \item 利用电荷观点 $W = \frac{1}{2} \int \rho V d\tau$,计算系统的总静电能 $W$。
    \item 利用场观点 $W = \frac{\epsilon_0}{2} \int E^2 d\tau$,计算系统的总静电能 $W$,并验证与 (a) 部的结果一致。
\end{enumerate}

\item 接地导电球外的点电荷
一个点电荷 $q$ 放置在距离半径为 $R$ 的接地导电球中心 $d$ 处 ($d>R$)。
\begin{enumerate}
    \item 利用镜像法,确定镜像电荷 $q'$ 的大小和位置 $d'$(相对于球心)。
    \item 求出球心到电荷 $q$ 之间连线上,球体表面上一点的电场强度 $\boldsymbol{E}$ 的大小。
    \item 求出电荷 $q$ 受到球体吸引力的方向和大小。
\end{enumerate}

\item 二维拉普拉斯方程的边值问题
考虑一个无限长的矩形区域(截面为 $a \times b$),三个侧面 ($x=0$, $x=a$, $y=0$) 均接地(电势 $V=0$)。第四个侧面 ($y=b$) 保持电势 $V(x, b) = V_0$(常数)。
\begin{enumerate}
    \item 写出在矩形区域内电势 $V(x, y)$ 满足的定解方程(包括微分方程和边界条件)。
    \item 利用分离变量法 $V(x, y) = X(x) Y(y)$,求出电势 $V(x, y)$ 的一般解。
    \item 确定最终解中级数的系数。
\end{enumerate}

\item 电偶极子系统的多极展开
一个电荷系统由三个点电荷组成:$q$ 位于 $(0, 0, a)$ 处,$-2q$ 位于原点 $(0, 0, 0)$ 处,另一个 $q$ 位于 $(0, 0, -a)$ 处。
\begin{enumerate}
    \item 计算该系统的总电荷(单极矩 $Q$)。
    \item 计算该系统的电偶极矩 $\boldsymbol{p}$。
    \item 写出在远处 ($r \gg a$) 电势 $V(\boldsymbol{r})$ 的最低阶非零近似项表达式。
\end{enumerate}


\item 均匀极化圆柱体的场
一个无限长、半径为 $R$ 的圆柱体被均匀极化,极化强度为 $\boldsymbol{P} = P_0 \hat{\boldsymbol{s}}$,其中 $\hat{\boldsymbol{s}}$ 是径向单位矢量(柱坐标)。周围是真空。
\begin{enumerate}
    \item 计算圆柱体的体束缚电荷密度 $\rho_b$ 和表面束缚电荷密度 $\sigma_b$。
    \item 利用束缚电荷法,求出圆柱体内部 ($s < R$) 的电场 $\boldsymbol{E}_{in}$。
    \item 求出圆柱体外部 ($s > R$) 的电场 $\boldsymbol{E}_{out}$。
\end{enumerate}

\item 同轴电容与分层电介质
一个无限长同轴电容器,内导体半径 $a$,外导体半径 $c$。内导体和外导体之间 ($a < s < c$) 填充了两种不同的线性电介质:区域 $a < s < b$ 填充了介质 $\epsilon_1$,区域 $b < s < c$ 填充了介质 $\epsilon_2$。假设内导体带有自由线电荷密度 $\lambda_f$。
\begin{enumerate}
    \item 利用 $\nabla \cdot \boldsymbol{D} = \rho_f$,求出两个区域内的辅助电场 $\boldsymbol{D}(s)$。
    \item 求出两个区域内的电场 $\boldsymbol{E}_1(s)$ 和 $\boldsymbol{E}_2(s)$。
    \item 求出在 $s=b$ 边界上的束缚面电荷密度 $\sigma_b$。
\end{enumerate}

\item 电介质中的泊松方程
一个无限大的线性电介质 ($\epsilon$) 内部,存在一个均匀分布的自由电荷球体,半径为 $R$,体自由电荷密度为 $\rho_f$。
\begin{enumerate}
    \item 写出电介质中,电势 $V$ 满足的微分方程(泊松方程)的修正形式。
    \item 利用高斯定律或泊松方程,求出介质内部 ($r < R$) 的电场 $\boldsymbol{E}_{in}$ 和电势 $V_{in}$。
    \item 求出介质外部 ($r > R$) 的电场 $\boldsymbol{E}_{out}$ 和电势 $V_{out}$。
\end{enumerate}

\item 一根无限长的圆柱形导线,半径为 $R$,沿 $z$ 轴放置。导线中流过总电流 $I$,电流密度 $\boldsymbol{J}$ 沿轴向 ($\hat{\boldsymbol{z}}$ 方向) 分布,其大小与柱坐标 $s$ (径向距离) 的平方成正比,即 $\boldsymbol{J}(s) = c s^2 \hat{\boldsymbol{z}}$,其中 $c$ 是常数。
\begin{enumerate}
    \item 求出常数 $c$ 与总电流 $I$ 和半径 $R$ 的关系。
    \item 利用安培定律,求出导线内部 ($s < R$) 的磁感应强度 $\boldsymbol{B}(s)$。
    \item 求出导线外部 ($s > R$) 的磁感应强度 $\boldsymbol{B}(s)$。
\end{enumerate}

\item 一个无限长、半径为 $R$ 的理想螺线管,单位长度匝数为 $n$,通有电流 $I$。螺线管沿 $z$ 轴放置。
\begin{enumerate}
    \item 利用安培定律确定螺线管内部和外部的磁场 $\boldsymbol{B}$。
    \item 在库仑规范 ($\nabla \cdot \boldsymbol{A} = 0$) 下,利用 $\boldsymbol{B} = \nabla \times \boldsymbol{A}$,求出螺线管内部 ($s < R$) 的磁矢势 $\boldsymbol{A}(s)$。
    \item 求出螺线管外部 ($s > R$) 的磁矢势 $\boldsymbol{A}(s)$。
\end{enumerate}

\item 一个无限长直导线沿 $x$ 轴放置,通有稳恒电流 $I$(沿 $+\hat{\boldsymbol{x}}$ 方向)。在 $y$ 轴上 $y=a$ 处放置一个微小磁偶极子 $\boldsymbol{m}$,其方向沿 $z$ 轴 ($\boldsymbol{m} = m \hat{\boldsymbol{z}}$)。
\begin{enumerate}
    \item 求出直导线在磁偶极子位置 ($y=a$) 产生的磁场 $\boldsymbol{B}(a)$。
    \item 利用公式 $\boldsymbol{F} = \nabla(\boldsymbol{m} \cdot \boldsymbol{B})$,计算直导线对磁偶极子 $\boldsymbol{m}$ 施加的净力 $\boldsymbol{F}$。
\end{enumerate}


\item 一个半径为 $R$ 的球体被均匀磁化,磁化强度为 $\boldsymbol{M} = M_0 \hat{\boldsymbol{z}}$。球体周围是真空。
\begin{enumerate}
    \item 计算球体的体束缚电流密度 $\boldsymbol{J}_b$ 和表面束缚电流密度 $\boldsymbol{K}_b$。
    \item 证明球体内部的磁感应强度 $\boldsymbol{B}_{in}$ 是均匀的,并求出其表达式。
    \item 求出球体外部($r>R$)的磁感应强度 $\boldsymbol{B}_{out}$。
\end{enumerate}

\item 考虑一个无限长同轴电缆,内导体内径为 $a$,外导体内径为 $b$,外径为 $c$。内导体和外导体之间 ($a < s < b$) 填充了一种线性磁介质,其相对磁导率为 $\mu_r$。内导体通有均匀分布的自由电流 $I$(沿 $z$ 轴正向),外导体通有 $-I$。
\begin{enumerate}
    \item 利用 $\nabla \times \boldsymbol{H} = \boldsymbol{J}_f$,求出区域 $a < s < b$ 内的辅助磁场 $\boldsymbol{H}$。
    \item 求出区域 $a < s < b$ 内的磁感应强度 $\boldsymbol{B}$ 和磁化强度 $\boldsymbol{M}$。
    \item 求出整个系统中所有的束缚电流密度 $\boldsymbol{J}_b$ 和 $\boldsymbol{K}_b$。
\end{enumerate}

\item 假设磁介质 1 (磁导率 $\mu_1$) 和磁介质 2 (磁导率 $\mu_2$) 之间存在一个平坦的交界面,界面上没有自由电流。
\begin{enumerate}
    \item 写出磁感应强度 $\boldsymbol{B}$ 和辅助磁场 $\boldsymbol{H}$ 在该交界面处的边界条件。
    \item 假设 $\boldsymbol{B}_1$ 与界面法线 $\hat{\boldsymbol{n}}$ (指向介质 2) 的夹角为 $\theta_1$。推导 $\boldsymbol{B}_1$ 与 $\boldsymbol{B}_2$ 之间的折射定律(即 $\theta_1$ 与 $\theta_2$ 之间的关系)。
\end{enumerate}
% Calculation 1: Resistance of Non-uniform Medium (Ex 7.2 or Prob 7.4 style)

\item 
一个矩形线圈(边长 $a$ 沿 $x$ 轴,边长 $b$ 沿 $y$ 轴)以恒定速度 $\boldsymbol{v} = v \hat{\boldsymbol{x}}$ 沿 $x$ 轴移动。线圈位于一个非均匀磁场中:
$$
\boldsymbol{B}(\boldsymbol{r}) = k x \hat{\boldsymbol{z}}
$$
其中 $k$ 是常数。在 $t=0$ 时刻,线圈的左边沿位于 $x=0$ 处。
\begin{enumerate}
    \item 计算 $t>0$ 时刻穿过线圈的总磁通量 $\Phi_B(t)$。
    \item 利用磁通量求导法($\mathcal{E} = - d\Phi_B/dt$)计算线圈中的感应电动势 $\mathcal{E}$。
    \item 利用动生电动势积分法($\mathcal{E} = \oint (\boldsymbol{v} \times \boldsymbol{B}) \cdot d\boldsymbol{l}$)验证 (b) 的结果。
\end{enumerate}

\item 
一个半径为 $R$ 的圆柱形电容器正在以恒定电流 $I$ 充电。假设两极板之间为理想介质(介电常数 $\epsilon$,电导率 $\sigma=0$)。
\begin{enumerate}
    \item 证明板间区域($s < R$)的位移电流密度 $\boldsymbol{J}_D$ 是均匀的,并写出 $\boldsymbol{J}_D$ 的大小(用 $I$ 和 $R$ 表示)。
    \item 在 $s < R$ 区域,计算 $\nabla \times \boldsymbol{B}$ 的大小(用 $\boldsymbol{J}_D$ 表示)。
    \item 利用斯托克斯定理 $\oint \boldsymbol{B} \cdot d\boldsymbol{l} = \mu_0 \int (\nabla \times \boldsymbol{H}) \cdot d\boldsymbol{a}$,计算在 $s < R$ 处的磁场大小 $B(s)$。
\end{enumerate}

\item 
一个具有电导率 $\sigma$ 和介电常数 $\epsilon$ 的无限大介质内部,在 $t=0$ 时刻注入了均匀的自由电荷密度 $\rho_0$。
\begin{enumerate}
    \item 从连续性方程和欧姆定律 $\boldsymbol{J} = \sigma \boldsymbol{E}$ 出发,推导出电荷密度 $\rho$ 满足的微分方程。
    \item 求解此微分方程,给出 $\rho(t)$ 的表达式。
    \item 计算铜($\sigma \approx 6 \times 10^7 \ \text{S/m}$,$\epsilon \approx \epsilon_0$)的弛豫时间 $\tau$,并说明在良导体中电荷的分布特征。
\end{enumerate}

\item 
两个共轴的圆形线圈 $C_1$ 和 $C_2$,半径分别为 $a$ 和 $b$(假设 $a \ll b$),相距 $z$。线圈 $C_1$ 中通有电流 $I_1$。
\begin{enumerate}
    \item 写出 $C_1$ 在其轴线上产生的磁场 $\boldsymbol{B}_1(z)$ 的近似表达式。
    \item 计算穿过线圈 $C_2$ 的磁通量 $\Phi_{21}$ 的近似值。
    \item 给出两个线圈之间的互感系数 $M_{21}$ 的近似表达式。
\end{enumerate}

\item 
在一个不存在自由电荷和传导电流的真空区域,存在一个电场 $\boldsymbol{E} = E_0 \sin(\omega t - k x) \hat{\boldsymbol{y}}$。
\begin{enumerate}
    \item 利用法拉第电磁感应定律 $\nabla \times \boldsymbol{E} = - \frac{\partial \boldsymbol{B}}{\partial t}$,求出相应的磁场 $\boldsymbol{B}(\boldsymbol{r}, t)$。
    \item 利用安培-麦克斯韦定律 $\nabla \times \boldsymbol{B} = \mu_0 \epsilon_0 \frac{\partial \boldsymbol{E}}{\partial t}$,验证所求 $\boldsymbol{E}$ 和 $\boldsymbol{B}$ 是否满足该方程。
\end{enumerate}

\item 
考虑两个线性均匀介质的交界面($z=0$),介质 1($z>0$)的参数为 $\epsilon_1, \sigma_1$,介质 2($z<0$)的参数为 $\epsilon_2, \sigma_2$。在交界面处没有自由面电荷。
\begin{enumerate}
    \item 写出非稳恒电流情况下,电位移矢量 $\boldsymbol{D}$ 的法向分量 $D_{1\perp}$ 和 $D_{2\perp}$ 之间满足的边界条件。
    \item 结合欧姆定律 $\boldsymbol{J} = \sigma \boldsymbol{E}$ 和边界条件,推导传导电流的法向分量 $J_{1\perp}$ 和 $J_{2\perp}$ 之间满足的关系。
    \item 假设交界面处存在一个稳定的自由面电荷密度 $\sigma_f$,这个稳定的 $\sigma_f$ 表达式是什么?
\end{enumerate}

\item 一根无限长、理想导体的同轴电缆,内导体内径为 $a$,外导体内径为 $b$。电缆两端接有电压为 $V$ 的电池(形成电场 $\boldsymbol{E}$)和负载电阻(消耗功率)。稳态时,内导体携带总电流 $I$,外导体携带回流电流 $-I$。忽略边缘效应。

\begin{enumerate}
    \item 求同轴电缆内 $a < s < b$ 区域的电场 $\boldsymbol{E}$ 和磁场 $\boldsymbol{B}$(以圆柱坐标系表示,$\hat{\boldsymbol{s}}$ 为径向,$\hat{\boldsymbol{\phi}}$ 为环向,$\hat{\boldsymbol{z}}$ 为轴向)。
    \item 计算 $a < s < b$ 区域的波印廷矢量 $\boldsymbol{S}$,并说明其方向所代表的物理意义。
    \item 通过积分 $\oint_{\mathcal{S}} \boldsymbol{S} \cdot d\boldsymbol{a}$ 证明通过电缆横截面的总功率 $P$ 等于 $VI$。
\end{enumerate}

\item 一个平行板电容器,极板面积为 $A$,间距为 $d$。两板之间的空间充满均匀电场 $\boldsymbol{E} = E_0 \hat{\boldsymbol{z}}$。此外,电容器置于一个均匀的外磁场 $\boldsymbol{B} = B_0 \hat{\boldsymbol{x}}$ 中。

\begin{enumerate}
    \item 写出该区域内电磁场动量密度 $\boldsymbol{g}$ 的表达式。
    \item 计算该区域内电磁场动量密度 $\boldsymbol{g}$ 的矢量方向和大小。
    \item 忽略边缘效应,求出电容器内储存的总电磁场动量 $\boldsymbol{P}_{em}$。
\end{enumerate}

\item 考虑一个带有均匀面电荷密度 $\sigma$ 的无限大平面导体薄板,其电场为 $\boldsymbol{E} = \frac{\sigma}{2\epsilon_0} \hat{\boldsymbol{z}}$(假设薄板位于 $z=0$ 平面,考虑 $z>0$ 区域)。现在考虑 $z>0$ 区域内的麦克斯韦应力张量 $\overleftrightarrow{T}$。

\begin{enumerate}
    \item 确定 $z>0$ 区域麦克斯韦应力张量 $\overleftrightarrow{T}$ 的所有分量,并以矩阵形式表示。
    \item 利用静电场中电磁力公式 $\boldsymbol{F} = \oint_{\mathcal{S}} \overleftrightarrow{T} \cdot d\boldsymbol{a}$,考虑一块位于 $z>0$ 区域、面积为 $A$ 的想象性闭合体元 $\mathcal{V}$(例如一个底面在 $z=0$ 上方的立方体),计算电场作用在底面上的力(即应力)的方向和大小。
    \item 基于应力张量的物理意义,解释导体薄板表面所受的净压力应如何计算。
\end{enumerate}

\item 一个在真空中传播的单色平面电磁波,已知其磁场为 $\boldsymbol{B}(\boldsymbol{r}, t) = B_0 \cos(\boldsymbol{k} \cdot \boldsymbol{r} - \omega t) \hat{\boldsymbol{y}}$。
\begin{enumerate}
    \item 写出该电磁波的波矢量 $\boldsymbol{k}$ 的一般表达式。
    \item 根据麦克斯韦方程组,求出对应的电场 $\boldsymbol{E}(\boldsymbol{r}, t)$ 的表达式。
\end{enumerate}


\item 电磁波从真空 ($\epsilon_0, \mu_0$) 垂直入射到一种非磁性介质 ($\mu=\mu_0, \epsilon=4\epsilon_0$)。
\begin{enumerate}
    \item 计算该介质的折射率 $n$ 和本征阻抗 $\eta$。
    \item 计算电场振幅的反射系数 $r$ 和透射系数 $t$。
    \item 确定反射波的强度 $R$ 和透射波的强度 $T$ (以入射波强度 $I$ 为单位),并验证 $R+T=1$。
\end{enumerate}


\item 海水是一种良导体,其电导率约为 $\sigma = 4 \text{ S/m}$,相对介电常数 $\epsilon_r \approx 80$,且 $\mu_r \approx 1$。
\begin{enumerate}
    \item 确定对于频率 $f = 100 \text{ Hz}$ 的无线电波,海水是良导体还是不良导体(通过比较位移电流和传导电流密度)。
    \item 计算该频率下电磁波在海水中的趋肤深度 $\delta$。
    \item 解释该结果对水下通信的影响。
\end{enumerate}


\item 光从介质 1 ($n_1 = 1.5$) 入射到介质 2 ($n_2 = 1.0$)。
\begin{enumerate}
    \item 计算两个介质之间的布儒斯特角 $\theta_B$(当 $E$ 场平行于入射面时,反射波为零的角度)。
    \item 计算全反射的临界角 $\theta_c$。
    \item 简述当入射角为 $45^\circ$ 时,介质 2 中是否存在传播波?
\end{enumerate}


\item 一个均匀的电磁波以强度 $I = 1000 \text{ W/m}^2$ 传播,作用在一个面积为 $A = 1 \text{ m}^2$ 的完全吸收表面上。
\begin{enumerate}
    \item 计算该电磁波的动量密度 $\boldsymbol{g}$ 的平均值 $\langle g \rangle$。
    \item 计算作用在该表面上的平均辐射压 $\langle P_{\text{rad}} \rangle$。
    \item 求出作用在该表面上的总平均电磁力 $\langle \boldsymbol{F} \rangle$。
\end{enumerate}


\item 一个理想矩形波导的横截面尺寸为 $a$ 和 $b$,其中 $a > b$。电磁波在其中传播的色散关系为 $\omega^2 = c^2 k^2 + \omega_c^2$。
\begin{enumerate}
    \item 写出 $\text{TE}_{10}$ 模式的截止频率 $\omega_c$ 表达式(以 $a$ 和 $c$ 表示)。
    \item 证明在该波导中,相速度 $v_p$ 总是大于光速 $c$。
    \item 证明群速度 $v_g$ 总是小于光速 $c$,并验证 $v_p v_g = c^2$。
\end{enumerate}


\item 给定某电磁场的一组势函数为 $V(\boldsymbol{r}, t) = 0$ 和 $\boldsymbol{A}(\boldsymbol{r}, t) = \frac{1}{2} \boldsymbol{B}_0 \times \boldsymbol{r} \sin(\omega t)$,其中 $\boldsymbol{B}_0$ 是常矢量。
\begin{enumerate}
    \item 判断该组势函数是否满足洛伦兹规范条件 $\nabla \cdot \boldsymbol{A} + \mu_0 \epsilon_0 \frac{\partial V}{\partial t} = 0$。
    \item 找到一个规范函数 $\lambda(\boldsymbol{r}, t)$,使得变换后的新标量势 $V'(\boldsymbol{r}, t)$ 满足 $V'(\boldsymbol{r}, t) = -(\boldsymbol{E}_0 \cdot \boldsymbol{r}) \cos(\omega t)$,其中 $\boldsymbol{E}_0$ 是另一个常矢量。
    \item 利用找到的 $\lambda(\boldsymbol{r}, t)$ 给出新的矢量势 $\boldsymbol{A}'(\boldsymbol{r}, t)$。
\end{enumerate}

\item 一个位于原点 ($\boldsymbol{r}'=0$) 的点电荷,其电荷量随时间作余弦振荡变化 $q(t) = q_0 \cos(\omega t)$。
\begin{enumerate}
    \item 写出该点电荷产生的延迟标量势 $V(\boldsymbol{r}, t)$ 的表达式。
    \item 计算该标量势 $V(\boldsymbol{r}, t)$ 对应的电场 $\boldsymbol{E}_V = -\nabla V$ 项(注意:$\nabla$ 是对 $\boldsymbol{r}$ 的求导,需要考虑 $\rho$ 对 $t_r$ 的依赖,其中 $t_r = t - r/c$)。
    \item 讨论在准静态近似 ($\omega \to 0$ 或 $c \to \infty$) 下,结果是否符合库仑定律。
\end{enumerate}

\item 一个点电荷 $q$ 沿着 $z$ 轴以恒定速度 $\boldsymbol{v} = v \hat{\boldsymbol{z}}$ 运动。在某一时刻 $t$ 观察到电荷位于 $z(t) = v t$ 处。
\begin{enumerate}
    \item 证明对于场点 $\boldsymbol{r}=(0, y, 0)$,其源点 $\boldsymbol{r}'(t_r) = (0, 0, z(t_r))$ 处的 Liénard-Wiechert 因子 $\mathcal{R} = R - \boldsymbol{R} \cdot \boldsymbol{v}/c$ 可以表示为:
    $$
    \mathcal{R} = \frac{R}{\gamma^2 (1 - v^2/c^2)}
    $$
    其中 $\gamma = 1/\sqrt{1 - v^2/c^2}$ 为洛伦兹因子,$R = |\boldsymbol{r} - \boldsymbol{r}'(t_r)|$。
    \item 利用勾股定理和延迟时间 $t_r$ 的定义 $R = c (t - t_r)$,求出在 $\boldsymbol{r}=(0, y, 0)$ 处观测时, $R^2$ 与 $y$ 和 $\gamma$ 的关系。
\end{enumerate}

 \item 一个电偶极子 $\boldsymbol{p}(t) = p_0 \cos(\omega t) \hat{\boldsymbol{z}}$ 位于原点。
\begin{enumerate}
    \item 写出在辐射区 ($kr \gg 1$) 产生的磁场 $\boldsymbol{B}$ 的表达式(只需写出振幅和相位因子)。
    \item 计算波印廷矢量 $\boldsymbol{S}$ 的时间平均值 $\langle \boldsymbol{S} \rangle$。
    \item 推导辐射的角分布 $dP/d\Omega$ 的表达式,并画出其在 $xz$ 平面上的示意图。
\end{enumerate}

 \item 一个电荷量为 $q$ 的粒子,在 $xy$ 平面内做半径为 $R$ 的匀速圆周运动,角速度为 $\omega$。
\begin{enumerate}
    \item 假设粒子是非相对论性的 ($v \ll c$),写出其瞬时辐射功率 $P(t)$ 的表达式。
    \item 计算其时间平均辐射功率 $\langle P \rangle$。
    \item 假设 $q = 1.6 \times 10^{-19} \text{ C}$(电子电荷),$R = 1 \text{ m}$,$\omega = 1 \text{ Hz}$,估算 $\langle P \rangle$ 的数量级,并说明电子做如此低速圆周运动时的辐射效应是否显著。
\end{enumerate}

 \item 对于一个局域的电荷-电流分布 $\rho(\boldsymbol{r}', t)$, $\boldsymbol{J}(\boldsymbol{r}', t)$,其产生的 $\boldsymbol{E}$ 场和 $\boldsymbol{B}$ 场可以分解为静场、感应场和辐射场。
\begin{enumerate}
    \item 在远场区,请简述电场 $\boldsymbol{E}$ 中随 $1/R^2$ 衰减的项(感应场)和随 $1/R$ 衰减的项(辐射场)的物理来源。
    \item 推导在远场区 $\boldsymbol{E}_{rad}$ 和 $\boldsymbol{B}_{rad}$ 之间的关系,并证明它们彼此垂直,且都垂直于传播方向 $\hat{\boldsymbol{r}}$。
\end{enumerate}

 \item 一个惯性系 $S'$ 以匀速 $\boldsymbol{v} = v \hat{\boldsymbol{x}}$ 相对于 $S$ 系运动。在 $S'$ 系中有一个均匀静电场 $\boldsymbol{E}' = E'_0 \hat{\boldsymbol{y}}$。
\begin{enumerate}
    \item 写出 $S'$ 系中的电磁场分量 $E'_x, E'_y, E'_z, B'_x, B'_y, B'_z$ 的值。
    \item 利用场变换公式($B'_x = B_x$ 等),计算 $S$ 系中的电场 $\boldsymbol{E}$ 和磁场 $\boldsymbol{B}$ 的分量。
    \item 讨论 $\boldsymbol{E}$ 场和 $\boldsymbol{B}$ 场在 $S$ 系中的相对方向关系。
\end{enumerate}

 \item 在洛伦兹规范下,四维势 $A^\mu = (V/c, \boldsymbol{A})$ 满足的方程为:
$$
\left( \nabla^2 - \frac{1}{c^2} \frac{\partial^2}{\partial t^2} \right) A^\mu = -\mu_0 J^\mu
$$
\begin{enumerate}
    \item 将左边的达朗贝尔算符 $\left( \nabla^2 - \frac{1}{c^2} \frac{\partial^2}{\partial t^2} \right)$ 用四维散度 $\partial_\mu$ 表示。
    \item 利用 $A^\mu$ 的四维形式和 $\partial^\mu = \frac{\partial}{\partial x^\mu}$,将上述方程写成完全协变形式,即只使用四维张量和四维矢量符号。
    \item 证明四维势方程的解 $A^\mu$ 的解可以写为延迟积分形式:
    $$
    A^\mu(\boldsymbol{r}, t) = \frac{\mu_0}{4\pi} \int \frac{J^\mu(\boldsymbol{r}', t_r)}{|\boldsymbol{r}-\boldsymbol{r}'|} d\tau'
    $$
    (只需说明推导思路和最终结果的物理意义,无需详细积分过程)。
\end{enumerate}

 \item 证明电荷守恒定律(连续性方程)$\frac{\partial \rho}{\partial t} + \nabla \cdot \boldsymbol{J} = 0$ 在四维表示下满足四维散度为零 $\partial_\mu J^\mu = 0$。

%
%  \item 计算标量函数 $T(x,y,z) = x^2 + 2y^2 + 3z^2$ 的梯度 $\nabla T$,并指出梯度的方向与函数值变化的关系。
%
%  \item 计算线积分 $\displaystyle \int_C \boldsymbol{v}\cdot d\boldsymbol{l}$,其中 $\boldsymbol{v} = y\hat{x} + x\hat{y}$,路径 $C$ 为 $x^2 + y^2 = 1$ 的圆周(逆时针方向,平面 $z=0$)。
%   \item 设有一均匀带电直线段,长度为 $2L$,线电荷密度为 $\lambda$。求其轴线上距离中心 $z$ 处的电场强度 $\boldsymbol{E}$,并讨论 $z \gg L$ 的极限情况。
%
%  \item 一球形体半径为 $R$,带均匀体电荷密度 $\rho$。利用高斯定律求球体内外的电场分布,并画出 $|E(r)|$ 随 $r$ 的变化曲线。
%
%  \item 一无限大平面带有均匀面电荷密度 $\sigma$。利用高斯定律求该平面两侧的电场强度与方向,并说明结果的物理意义。
%
%  
%    \item 一半径为 $R$ 的均匀极化球体,其极化矢量为常量 $\boldsymbol{P}=P_0\hat{z}$。求:
%  \begin{enumerate}
%    \item 束缚面电荷密度 $\sigma_b$;
%    \item 球内、球外的电场分布。
%  \end{enumerate}
%  
%    \item 一根无限长圆柱介质,半径 $a$,极化为 $\boldsymbol{P}=P_0\hat{x}$。求柱体内外的电场。
%
%  \item 平行板电容器间距为 $d$,内充满线性介质,介电常数为 $\varepsilon_r$,上、下板自由电荷密度为 $\pm\sigma$。求:
%  \begin{enumerate}
%    \item $\boldsymbol{E}$、$\boldsymbol{D}$、$\boldsymbol{P}$;
%    \item 电位差 $V$;
%    \item 电容 $C$。
%  \end{enumerate}
%  
%  \item 计算:一长直导线载有电流 $I$,求距离导线 $s$ 处磁场 $\boldsymbol{B}$ 的大小和方向。
%  \item 一圆形电流环(半径 $R$)载流 $I$,求其轴线上距中心 $z$ 处磁场。
%
%  \item 一无限长螺线管单位长度有 $n$ 匝电流环,每匝电流为 $I$,求其内部和外部磁场。
%\item 计算:一无限长圆柱体具有均匀磁化 $\boldsymbol{M} = M_0 \hat{z}$。求其内部和外部磁场分布。
%\item 一个无限长螺线管(单位长度匝数 $n$,电流 $I$),内部填充磁化率为 $\chi_m$ 的线性材料。求内部磁场。
%\item 一个线性磁介质的球体置于外部均匀磁场 $\boldsymbol{B}_0$ 中,求球内磁场。
%
%  
\end{enumerate}

\newpage


\section*{矢量微分}

\subsection*{直角坐标系}
\[
d\boldsymbol{l} = dx\,\hat{\boldsymbol{x}} + dy\,\hat{\boldsymbol{y}} + dz\,\hat{\boldsymbol{z}}, \quad d\tau = dx\,dy\,dz
\]

梯度: $\quad\nabla t = \frac{\partial t}{\partial x}\hat{\boldsymbol{x}} + \frac{\partial t}{\partial y}\hat{\boldsymbol{y}} + \frac{\partial t}{\partial z}\hat{\boldsymbol{z}}$

散度: $\quad\nabla \cdot \boldsymbol{v} = \frac{\partial v_x}{\partial x} + \frac{\partial v_y}{\partial y} + \frac{\partial v_z}{\partial z}$

旋度: $\quad\nabla \times \boldsymbol{v} = \left(\frac{\partial v_z}{\partial y} - \frac{\partial v_y}{\partial z}\right)\hat{\boldsymbol{x}} + \left(\frac{\partial v_x}{\partial z} - \frac{\partial v_z}{\partial x}\right)\hat{\boldsymbol{y}} + \left(\frac{\partial v_y}{\partial x} - \frac{\partial v_x}{\partial y}\right)\hat{\boldsymbol{z}}$

拉普拉斯算子: $\quad\nabla^2 t = \frac{\partial^2 t}{\partial x^2} + \frac{\partial^2 t}{\partial y^2} + \frac{\partial^2 t}{\partial z^2}$

\subsection*{球坐标系}
\[
d\boldsymbol{l} = dr\,\hat{\boldsymbol{r}} + r\,d\theta\,\hat{\boldsymbol{\theta}} + r\sin\theta\,d\phi\,\hat{\boldsymbol{\phi}}, \quad d\tau = r^2\sin\theta\,dr\,d\theta\,d\phi
\]

梯度: $\quad\nabla t = \frac{\partial t}{\partial r}\hat{\boldsymbol{r}} + \frac{1}{r}\frac{\partial t}{\partial\theta}\hat{\boldsymbol{\theta}} + \frac{1}{r\sin\theta}\frac{\partial t}{\partial\phi}\hat{\boldsymbol{\phi}}$

散度: $\quad\nabla \cdot \boldsymbol{v} = \frac{1}{r^2}\frac{\partial}{\partial r}(r^2 v_r) + \frac{1}{r\sin\theta}\frac{\partial}{\partial\theta}(\sin\theta\,v_\theta) + \frac{1}{r\sin\theta}\frac{\partial v_\phi}{\partial\phi}$

旋度:
\begin{align*}
\nabla \times \boldsymbol{v} &= \frac{1}{r\sin\theta}\left[\frac{\partial}{\partial\theta}(\sin\theta\,v_\phi) - \frac{\partial v_\theta}{\partial\phi}\right]\hat{\boldsymbol{r}} \\
&\quad + \frac{1}{r}\left[\frac{1}{\sin\theta}\frac{\partial v_r}{\partial\phi} - \frac{\partial}{\partial r}(r v_\phi)\right]\hat{\boldsymbol{\theta}} \\
&\quad + \frac{1}{r}\left[\frac{\partial}{\partial r}(r v_\theta) - \frac{\partial v_r}{\partial\theta}\right]\hat{\boldsymbol{\phi}}
\end{align*}

拉普拉斯算子: $\quad\nabla^2 t = \frac{1}{r^2}\frac{\partial}{\partial r}\left(r^2\frac{\partial t}{\partial r}\right) + \frac{1}{r^2\sin\theta}\frac{\partial}{\partial\theta}\left(\sin\theta\frac{\partial t}{\partial\theta}\right) + \frac{1}{r^2\sin^2\theta}\frac{\partial^2 t}{\partial\phi^2}$

\subsection*{柱坐标系}
\[
d\boldsymbol{l} = ds\,\hat{\boldsymbol{s}} + s\,d\phi\,\hat{\boldsymbol{\phi}} + dz\,\hat{\boldsymbol{z}}, \quad d\tau = s\,ds\,d\phi\,dz
\]

梯度: $\quad\nabla t = \frac{\partial t}{\partial s}\hat{\boldsymbol{s}} + \frac{1}{s}\frac{\partial t}{\partial\phi}\hat{\boldsymbol{\phi}} + \frac{\partial t}{\partial z}\hat{\boldsymbol{z}}$

旋度: $\quad\nabla \cdot \boldsymbol{v} = \frac{1}{s}\frac{\partial}{\partial s}(s v_s) + \frac{1}{s}\frac{\partial v_\phi}{\partial\phi} + \frac{\partial v_z}{\partial z}$

旋度:
\begin{align*}
\nabla \times \boldsymbol{v} &= \left[\frac{1}{s}\frac{\partial v_z}{\partial\phi} - \frac{\partial v_\phi}{\partial z}\right]\hat{\boldsymbol{s}} + \left[\frac{\partial v_s}{\partial z} - \frac{\partial v_z}{\partial s}\right]\hat{\boldsymbol{\phi}} \\
&\quad + \frac{1}{s}\left[\frac{\partial}{\partial s}(s v_\phi) - \frac{\partial v_s}{\partial\phi}\right]\hat{\boldsymbol{z}}
\end{align*}

拉普拉斯算子: $\quad\nabla^2 t = \frac{1}{s}\frac{\partial}{\partial s}\left(s\frac{\partial t}{\partial s}\right) + \frac{1}{s^2}\frac{\partial^2 t}{\partial\phi^2} + \frac{\partial^2 t}{\partial z^2}$

\newpage

\section*{矢量恒等式}

\subsection*{三重积}
\begin{enumerate}
\item $\boldsymbol{A} \cdot (\boldsymbol{B} \times \boldsymbol{C}) = \boldsymbol{B} \cdot (\boldsymbol{C} \times \boldsymbol{A}) = \boldsymbol{C} \cdot (\boldsymbol{A} \times \boldsymbol{B})$
\item $\boldsymbol{A} \times (\boldsymbol{B} \times \boldsymbol{C}) = \boldsymbol{B}(\boldsymbol{A} \cdot \boldsymbol{C}) - \boldsymbol{C}(\boldsymbol{A} \cdot \boldsymbol{B})$
\end{enumerate}

\subsection*{积规则}
\begin{enumerate}
\setcounter{enumi}{2}
\item $\nabla(fg) = f(\nabla g) + g(\nabla f)$
\item $\nabla(\boldsymbol{A} \cdot \boldsymbol{B}) = \boldsymbol{A} \times (\nabla \times \boldsymbol{B}) + \boldsymbol{B} \times (\nabla \times \boldsymbol{A}) + (\boldsymbol{A} \cdot \nabla)\boldsymbol{B} + (\boldsymbol{B} \cdot \nabla)\boldsymbol{A}$
\item $\nabla \cdot (f\boldsymbol{A}) = f(\nabla \cdot \boldsymbol{A}) + \boldsymbol{A} \cdot (\nabla f)$
\item $\nabla \cdot (\boldsymbol{A} \times \boldsymbol{B}) = \boldsymbol{B} \cdot (\nabla \times \boldsymbol{A}) - \boldsymbol{A} \cdot (\nabla \times \boldsymbol{B})$
\item $\nabla \times (f\boldsymbol{A}) = f(\nabla \times \boldsymbol{A}) - \boldsymbol{A} \times (\nabla f)$
\item $\nabla \times (\boldsymbol{A} \times \boldsymbol{B}) = (\boldsymbol{B} \cdot \nabla)\boldsymbol{A} - (\boldsymbol{A} \cdot \nabla)\boldsymbol{B} + \boldsymbol{A}(\nabla \cdot \boldsymbol{B}) - \boldsymbol{B}(\nabla \cdot \boldsymbol{A})$
\end{enumerate}

\subsection*{二阶导数}
\begin{enumerate}
\setcounter{enumi}{8}
\item $\nabla \cdot (\nabla \times \boldsymbol{A}) = 0$
\item $\nabla \times (\nabla f) = 0$
\item $\nabla \times (\nabla \times \boldsymbol{A}) = \nabla(\nabla \cdot \boldsymbol{A}) - \nabla^2 \boldsymbol{A}$
\end{enumerate}

\section*{基本定理}

梯度定理: $\quad\int_a^b (\nabla f) \cdot d\boldsymbol{l} = f(b) - f(a)$

散度定理: $\quad\int (\nabla \cdot \boldsymbol{A})\, d\tau = \oint \boldsymbol{A} \cdot d\boldsymbol{a}$

旋度定理: $\quad\int (\nabla \times \boldsymbol{A}) \cdot d\boldsymbol{a} = \oint \boldsymbol{A} \cdot d\boldsymbol{l}$
\end{document}
%%%%%%%%%%%%%%%%%%%%%%%%%%%%%%%%%%%%%%%%%%%%%%%%%%%%%%%%%%%%%%%%%%%%%%%%%%%%%%%%%%%%%%%%%%%%%%%%%%%%
